\section{Минимизация максимального взвешенного временн\'{о}го смещения (ВВС) выполнения заказов на доставку груза
  для двух станций}
\subsection{Введение}\label{rzd:sec:int:1}
Рассматриваются следующие задачи.\\
\begin{problem}\label{rzd:pr:1}
Минимизация максимального взвешенного временн\'{о}го смещения (ВВС) выполнения заказов
  для двух станций с фиксированными моментами отправления поездов.
\end{problem}

Имеются две станции, соединенные двухпутной железной
дорогой. Необходимо выполнить множества заказов
$N^1 = \{O_1^1,\dots,O_n^1\}$ и $N^2=\{O_1^2,\dots,O_m^2\}$ на поставку
грузов между станциями. Заказы множества $N^1$ необходимо доставить с
первой станции на вторую, а заказы множества $N^2$ со второй на
первую. Каждый заказ состоит из одного вагона. Так как железная дорога двухпутная, то расписания для каждого из направлений движения можно составлять отдельно, следовательно для решения задачи достаточно последовательно составить расписания для множеств $N^1$ и $N^2$. Рассматриваемое множество заказов, без ограничения общности, обозначим через $N = \{O_1, \dots, O_n\}$. Для каждого заказа $O_j \in N$ определены следующие параметры: $r_j$ --- \textit{время поступления} заказа $O_j$, на станцию отправления, $w_j > 0$ --- \textit{ценность} (вес, важность) заказа. Для каждого заказа определен \textit{директивный срок} $d_j$ --- момент времени, до которого заказ желательно доставить на станцию назначения. Без ограничения общности будем считать, что заказы пронумерованы по возрастанию моментов поступления, $r_1 \leq r_2 \leq \dots \leq r_n$.

Доставка заказов с одной станции на другую осуществляется $q$ поездами, состоящими из $k_1, \dots, k_q$, вагонов, $\sum\limits_{i=1}^q k_i = n$, и отправляющимися в фиксированные моменты времени $S^1, \dots, S^q$. Не ограничивая общности, будем полагать что $S^1< \dots < S^q$. Скорость движения поездов может быть различной. \textit{Расписанием} $\pi$ будем называть последовательность пар $ \pi = \{(B^1, S^1), \dots, (B^q, S^q)\},$ где $B^i(\pi)$ --- множество заказов, перевозимых поездом $i$, $i=1, \dots, q,$ при расписании $\pi$, а $S^i(\pi)$ --- время отправления поезда $i$. Расписание $\pi$ будем называть \textit{допустимым}, если для любого поезда $i = 1, \dots, q$ и любого заказа $O_j \in B^i(\pi)$ выполнено
$$r_j \leq S^i(\pi),$$
т.е. момент поступления заказа $O_j$ на станцию отправления не превосходит момента отправления поезда, в который он включён. Множество всех допустимых расписаний обозначим через $\Pi(N)$. Таким образом, при формировании $i$-ого поезда можно рассматривать только множество заказов $N_i \subseteq N$, поступившиx к моменту времени $S^i$.

Момент времени $C^i$ прибытия поезда $i$, $i=1, \dots, q,$ на станцию назначения является фиксированным.
Поезд не может догнать или обогнать поезд, который отправился раньше него, т.е. для всех $i=1, \dots, q-1$ и $l=2, \dots, q$, таких, что $i<l$, будет выполнено неравенство:
\begin{equation*}
C^i < C^l.
\end{equation*}

Пусть $C_j(\pi)$ --- фактический момент времени доставки заказа $O_j \in N$ на станцию назначения при расписании $\pi$. Заметим, что $C_j(\pi)$ --- момент прибытия заказа, а $C^i$ --- момент прибытия поезда на станцию назначения. \textit{Временн\'{о}е смещение} заказа $O_j$ при расписании $\pi$ может быть вычислено по формуле:
$$L_j(\pi) = C_j(\pi) - d_j.$$
Требуется найти расписание $\pi$ с минимальным значением максимального взвешенного временн\'{о}го смещения (ВВС):
\begin{equation*}
  \min\limits_{\pi \in \Pi(N)}\max\limits_{O_j\in N}w_j L_j(\pi).
\end{equation*}

Данную задачу можно обозначить следующим образом $RS2|r_j, S^i|wL_{max},$ как это принято в теории расписаний \cite{GLLR:79}. Далее в статье в части \ref{rzd:sec:aux1:2} для данной задачи сформулирована вспомагательная задача, а также предложен алгоритм её решения. Алгоритм построения оптимального решения задачи \ref{rzd:pr:1} представлен в части \ref{rzd:sec:main1:3}.\\

\begin{problem}{Минимизация максимального взвешенного временн\'{о}го смещения (ВВС) выполнения заказов для двух станций.}\label{rzd:pr:2}
\end{problem}

Рассмотрим задачу, когда моменты отправления поездов не являются фиксированными значениями, а моменты их прибытия на станцию назначения задаются как функции от момента отправления.

\begin{figure}[h!]
\center{\includegraphics[width=16cm]{rzd_new_1.png}}
\caption{Условные обозначения задачи \ref{rzd:pr:2}.}
\label{rzd:pic:1}
\end{figure}

Характеристики заказов задаются параметрами, аналогичными рассмотренным ранее. Доставка заказов с одной станции на другую осуществляется $q$ поездами, состоящими из $k_1, \dots, k_q $, вагонов, также предполагается, что $\sum\limits_{i=1}^q k_i = n$.

\textit{Расписанием} $\pi$ будем называть последовательность пар $\pi = \{(B^1(\pi), S^1(\pi)), \dots, (B^q(\pi), S^q(\pi))\},$ где $B^i(\pi)$ --- множество заказов, перевозимых поездом $i$, $i=1, \dots, q,$ а $S^i(\pi)$ --- время отправления поезда $i$ при расписании $\pi$. Скорость движения поездов может быть различной. В случае, когда понятно о каком расписании идёт речь, моменты отправления и множество заказов, перевозимых поездом $i$, будем обозначать через $S^i$ и $B^i$ соответственно.

Обозначим через $\widehat{u}$ время, которое должно разделять моменты отправления двух поездов. Движение пассажирских поездов и электричек, плановые ремонтные работы и другие ограничения учитываются с помощью множества \textit{допустимых интервалов отправления} $U = \{[u_1^s, u_1^f), \dots, [u_v^s, u_v^f)\},$ где $v$ --- общее количество интервалов, когда может быть осуществлено отправления поезда (рисунок \ref{rzd:pic:1}). Расписание $\pi$ будем называть \textit{допустимым}, если момент отправления каждого поезда принадлежит одному из допустимых интервалов отправления, т.е. для любого $i = 1, \dots, q$ выполнено
$$S^i(\pi) \in U.$$
Множество всех допустимых расписаний обозначим через $\Pi(N, U)$. Время прибытия $i$-ого поезда, $i=1, \dots, q$, на станцию назначения может быть вычислено по формуле
$$C^i(\pi) = f_i(S^i(\pi)),$$
где $f_i(t)$ --- монотонно неубывающая функция от времени $t$. При любом расписании $\pi$ поезд не может догнать или обогнать поезд, который отправился раньше него, т.е. для всех $i=1, \dots, q-1$ и $l=2, \dots, q$, таких, что $i<l$, будет выполнено неравенство:
\begin{equation}\label{rzd:eq:0}
f_i(S^i(\pi)) < f_l(S^l(\pi)).
\end{equation}
Пусть $C_j(\pi)$ --- момент времени доставки заказа $O_j \in N$ на станцию назначения при расписании $\pi$. Временн\'{о}е смещение заказа $O_j$ при расписании $\pi$ может быть вычислено по формуле:
$$L_j(\pi) = C_j(\pi) - d_j.$$
Требуется найти расписание $\pi$ с минимальным значением максимального взвешенного временн\'{о}го смещения (ВВС):
\begin{equation*}
  \min\limits_{\pi \in \Pi(N,U)}\max\limits_{O_j\in N}w_j L_j(\pi).
\end{equation*}
Так как количество допустимых расписаний, при данной целевой функции может быть больше одного, выберем такое расписание, при котором время доставки всех заказов минимально. Пусть
$$y^* = \min\limits_{\pi \in \Pi(N,U)}\max\limits_{O_j\in N} w_j L_j(\pi).$$
Таким образом требуется построить расписание оптимальное по критерию:
\begin{equation}\label{rzd:eq:1}
  \min\limits_{\pi \in \Pi(N,U)} \max\limits_{O_j\in N} C_j(\pi) | \max\limits_{O_j \in N} w_j L_j(\pi) = y^*.
\end{equation}

Данную задачу можно обозначить как $RS2|r_j|C_{max}, wL_{max}$. В части \ref{rzd:sec:aux2:4} статьи для данной задачи сформулирована вспомогательная задачаи рассмотрены некоторые её свойства. Алгоритм решения вспомогательной задачи представлен в части \ref{rzd:sec:aux2c:6}. В части \ref{rzd:sec:main2:7} статьи предложен алгоритм построения оптимального решения задачи \ref{rzd:pr:2}. Анализ трудоёмкости алгоритма решения задачи \ref{rzd:pr:2} представлен в части \ref{rzd:sec:comp:8}.

Задачи оптимизации железнодорожных грузоперевозок являются объектом исследования многих ведущих специалистов и научных групп. Обзор некоторых ранее полученных результатов по данной теме представлен в работах \cite{CTV:2000} и \cite{dO:2001}. Некоторые оптимизационные задачи и их практические приложения для железных дорог Германии и Франции были рассмотрены в работах \cite{BHK:2002} и \cite{S:2011}, соответственно. В работах \cite{GDL:2015} и \cite{DKL:2014} рассмотрена задача оптимизации грузоперевозок между двумя железнодорожными станциями для однопутной железной дороги, одинаковой скорости движения поездов и различных критериев оптимальности.

Данная статья является продолжением работ, представленных в докладах \cite{LMA:2012}, \cite{AL:2012}, \cite{AL:2014}. В отличие от ранее полученных результатов, в данной статье рассматривается постановка задачи с различным количеством вагонов в поездах и нефиксированным временем движения поездов между станциями.


\subsection{Задача нахождения расписания с ограничением на максимальное ВВС}\label{rzd:sec:aux1:2}
Сформулируем вспомогательную задачу для задачи \ref{rzd:pr:1}.
Пусть задано ограничение на ВВС $y$. Расписание $\pi \in \Pi(N)$ будем называть \textit{разрешённым,} если для него выполняется неравенство
$$\max\limits_{O_j\in N}w_j L_j(\pi) < y.$$
Множество разрешённых расписаний для заданного ограничения $y$ будем обозначать через $\Phi(N, y)$. Требуется построить расписание $\pi \in \Phi(N, y) \subseteq \Pi(N)$ или доказать что $\Phi(N, y) = \emptyset$.

Заметим, что заказ $O_j \in N$ не будет нарушать ограничение $y$ тогда, когда он принадлежит составу $B^i$, $i=1, \dots, q$, и выполнено неравенство
$$C^i < d_j + \frac{y}{w_j},$$
т.е. момент прибытия поезда на станцию назначения меньше чем момент времени $d_j + \frac{y}{w_j}$. Если бы заказ $O_j$ прибыл на станцию назначения в момент времени $t$, такой что
$$t \geq d_j + \frac{y}{w_j},$$
то ВВС заказа $O_j$ было бы равно
$$(t - d_j)w_j \geq (d_j + \frac{y}{w_j} - d_j)w_j = y.$$
Таким образом, ограничение на ВВС было бы нарушено.

Для решения данной задачи предложим следующий алгоритм.

На первом шаге для каждого требования $O_j \in N$ найдём поезд с наибольшим номером $T^{max}(j, y)$, в который может быть включён данный заказ, не нарушив ограничения на ВВС $y$:
$$T^{max}(j, y) = \arg \max\limits_{i = 1, \dots, q} C^i | C^i < d_j + \frac{y}{w_j}.$$

Далее будем последовательно формировать множества (составы) $B^1(\pi(y)), B^2(\pi(y)), \dots, B^q(\pi(y))$ по следующему правилу. В состав $B^i(\pi(y))$, $i=1, \dots, q$ могут быть включены только заказы из $N_i$, где $N_i \subseteq N$ --- множество заказов, поступившиx к моменту времени $S^i$. Множество $B^i(\pi(y))$, $i=1, \dots, q$, состоит из $k_i$ заказов $O_{i_1}(\pi(y)), \dots, O_{i_{k_i}}(\pi(y)) \in N_i \setminus\{B^1(\pi(y)), \dots, B^{i-1}(\pi(y))\}$, для которых выполнено:
$$T^{max}(i_1,y) \leq T^{max}(i_2,y) \leq \dots \leq T^{max}(i_{k_i},y),$$
и для любого $O_j \in N_i\setminus\{B^1(\pi(y)), \dots, B^{i}(\pi(y))\}$ выполняется неравенство
$$T^{max}(i_{k_i},y) \leq T^{max}(j,y).$$
Таким образом, в формируемый состав будут включено $k_i$ заказов с наименьшими значениями $T^{max}(j,y)$, не принадлежащие ранее сформированным составам.

В случае, если при формировании $i$-ого состава для некоторого заказа $O_{j}(\pi(y)) \in B^i(\pi(y))$ выполнено
$$T^{max}(j,y) < i,$$
то заказ $O_j$ нарушит ограничение на ВВС. Следовательно не существует расписания $\pi(y) \in \Phi(N, y)$, удовлетворяющего ограничению $y$, после чего алгоритм завершает работу.

Если при формировании поезда $i$, $i=1, \dots, q$, количество заказов, поступивших до момента отправления, недостаточно, т.е.
$$|N_i \setminus \{B^1(\pi(y)), \dots, B^{i-1}(\pi)\}| < k_i,$$
то, из того, что $|B^1(\pi(y))| = k_1$, $|B^2(\pi(y))| = k_2$, \dots, $|B^{i-1}(\pi(y))| = k_{i-1}$, следует, что
$$|N_i| < \sum\limits_{l=1}^{i} k_l,$$
т.е. невозможно сформировать состав $i$.
Таким образом, допустимых расписаний не существует и $\Pi(N) = \emptyset$, а значит и $\Phi(N, y) = \emptyset$. В этом случае алгоритм завершает работу. Таким образом, алгоритм заканчивает работу, когда построено допустимое расписание, либо такого расписания невозможно построить.

%Формальная запись алгоритма выглядит следующим образом:
\begin{algorithm}[H]\label{rzd:alg:0}
\NoCaptionOfAlgo
\caption{\textbf{Алгоритм \ref{rzd:alg:0}}}
\begin{itemize}
\item[0.] Входные данные:\\
 $N, q, y, k_1, \dots, k_q, S^1, \dots, S^q, C^1, \dots, C^q.$

\item[1.] Для всех требований $O_j$ вычисляются значения $$T^{max}(j, y) = \arg \max\limits_{i = 1, \dots, q} C^i | C^i < d_j + \frac{y}{w_j}.$$ Также выполняется включение требований $O_j$, $j=1, \dots, n,$ во множества $N_i$, $i=1, \dots, q,$ по правилу:
    $$O_j \in N_i \Leftrightarrow S^i \geq r_j.$$
\item[2.] Если
          $$|N_i \setminus \{B^1(\pi),\dots, B^{i-1}(\pi)\}| < k_i,$$
           то выполняется присвоение $\pi(y) = \emptyset$ и алгоритм завершает свою работу. Иначе, переходим на шаг 3.
\item[3.] Включаем во множество $B^i(\pi)$ $k_i$ заказов из множества $N_i \setminus \{B^1(\pi),\dots, B^{i-1}(\pi)\}$ с наименьшими значениями $T^{max}(i_1,y), \dots, T^{max}(i_{k_i},y)$. Если для некоторого заказа $O_j \in B^i(\pi)$ выполнено
    $$T^{max}(j,y) < i,$$
    то выполняется присвоение $\pi(y) = \emptyset$ и алгоритм завершает работу. Иначе, переходим на шаг 4.
\item[4.] Добавляем сформированный состав к расписанию $\pi(y)$:
$$\pi(y) := \pi(y) \cup \{B^i(\pi), S^i\}.$$
\item[5.] Если $i \neq q$, то выполняется присвоение $i:=i+1$ и переход на шаг 2. Иначе, расписание $\pi(y)$ построено и алгоритм завершает работу.
\end{itemize}
\end{algorithm}

\begin{theorem}\label{rzd:th:0}
Если в результате работы алгоритма \ref{rzd:alg:0} построено расписание $\pi(y)= \emptyset,$ то расписаний, удовлетворяющих ограничению $y$ для данного набора требований, не существует, т.е. $\Phi(N,y) = \emptyset$.

Если $\Phi(N, y) \neq \emptyset$, то в результате работы алгоритма \ref{rzd:alg:0} будет построено расписание $\pi(y) \in \Phi(N,y)$, удовлетворяющее неравенству
$$\max\limits_{O_j\in N}w_j L_j(\pi(y)) < y.$$
\end{theorem}
\begin{proof}
Пусть в результате выполнения алгоритма \ref{rzd:alg:0} построено расписание $\pi(y) = \emptyset$.
Предположим, что алгоритм остановился при формировании $i$-ого поезда. Если алгоритм остановился на шаге 2, то выполняется неравенство
$$|N_i \setminus \{B^1(\pi(y)),\dots, B^{i-1}(\pi(y))\}| < k_i,$$
а значит допустимых расписаний не существует и $\Pi(N) = \emptyset$. Следовательно $\Phi(N, y) \subseteq \Pi(N) = \emptyset.$

Если алгоритм остановился на шаге 3, то из алгоритма \ref{rzd:alg:0} следует, что при проходе шагов 2-5 алгоритма формирования поездов с номерами $l = 1, 2, \dots, i-1$ на шаге 3 в состав $B^l(\pi(y))$ включалось максимально возможное количество заказов
среди поступивших к моменту $S^l$, для которых выполняется неравенство
$T^{max}({j},y) < i.$
Таким образом, к моменту отправления $i$-ого поезда $S^i$ поступило меньше чем $\sum\limits_{l=1}^i k_l$ заказов $j$, для которых выполнено
$$T^{max}({j},y) \geq i,$$
следовательно, $\Phi(N,y) = \emptyset.$

Если же алгоритм построил расписание $\pi(y),$ то т.к. при нём любой заказ $O_j \in N$ отправляется поездом, с номером не большим чем $T^{max}(j,y),$ то расписание $\pi(y) \in \Phi(N,y)$ удовлетворяет ограничению
$$\max\limits_{O_j\in N}w_j L_j(\pi(y)) < y.$$
Теорема доказана.
\end{proof}
\begin{lemma}\label{rzd:lm:0}
Трудоёмкость алгоритма \ref{rzd:alg:0} решения вспомогательной задачи $RS2|r_i, S^j|wL_{max} < y$ составляет $O(n \log n)$ операций, где $n$ ---количество заказов.
\end{lemma}
\begin{proof}
Наиболее трудоёмкой частью алгоритма \ref{rzd:alg:0} является вычисление и сортировка $n$ значений $T^{max}(1, y), \dots, T^{max}(n, y)$ на шаге 1. Следовательно, трудоёмкость не превышает $O(n \log n)$ операций.
\end{proof}

\subsection{Алгоритм решения задачи минимизации ВВС при фиксированных моментах отправления поездов}\label{rzd:sec:main1:3}
Для нахождения оптимального расписания по критерию
\begin{equation*}
  \min\limits_{\pi \in \Pi(N)}\max\limits_{O_j\in N}w_j L_j(\pi).
\end{equation*}
будем действовать следующим образом.

На первом шаге с помощью алгоритма \ref{rzd:alg:0} построим расписание $\pi_1 = \pi(y_1)$ для значения ограничения $y_1 = +\infty$ и вычислим значение $wL_{max}(\pi_1)$. Затем будем строить расписания $\pi_2 = \pi(y_2)$ для значения ограничения $y_2 = wL_{max}(\pi_1)$. Последовательно строим расписания $\pi_3 =\pi(y_3)$, $\dots$, $\pi_{l} =\pi(y_l)$, где $y_i = wL_{max} (\pi_{i-1})$, $i=3, \dots, l$, до тех пор, пока алгоритм \ref{rzd:alg:0} не завершится с результатом $\pi(wL_{max}(\pi_{l})) = \emptyset.$ В этом случае работа алгоритма \ref{rzd:alg:0,5} завершается, результатом работы является расписание $\pi^{*}(N) = \pi_{l}$, построенное на предыдущем шаге. Далее будет показано, что количество построенных расписаний $\pi_1, \dots, \pi_l$ не превосходит $n(q-1)$.

%Формальная запись алгоритма \ref{rzd:alg:0,5} выглядит следующим образом:
\begin{algorithm}[H]\label{rzd:alg:0,5}
\NoCaptionOfAlgo
\caption{\textbf{Алгоритм \ref{rzd:alg:0,5}}}
\begin{itemize}
\item[0.] Входные данные:\\
$N, q, y_1 = +\infty, i=1, k_1, \dots, k_q, S^1, \dots, S^q, C^1, \dots, C^q. $
\item[1.] Выполняется построение расписания $\pi_i = \pi(y_i)$ с помощью алгоритма \ref{rzd:alg:0}.
\item[2.] Если $\pi_i \neq \emptyset,$ то
    \begin{itemize}
    \item[a)] присваиваем $\pi^{*}(N) = \pi_i$;
    \item[b)] присваиваем $y_{i+1} = wL_{max}(\pi_i),$ $i:=i+1,$ и переходим на шаг 1.
    \end{itemize}
\item[3.] Если $\pi_i = \emptyset$, алгоритм возвращает расписание $\pi^{*}(N)$ и завершает работу.
\end{itemize}
\end{algorithm}

\begin{theorem} \label{rzd:th:0,5}
Расписание, построенное в результате работы алгоритма \ref{rzd:alg:0,5} является оптимальным по критерию
\begin{equation*}
  \min\limits_{\pi \in \Pi(N)}\max\limits_{O_j\in N}w_j L_j(\pi).
\end{equation*}
Если для расписания $\pi^{*}(N)$ выполняется
$$\pi^{*}(N) = \emptyset,$$
то $\Pi(N) = \emptyset,$ т.е. допустимого расписания не существует.
\end{theorem}
\begin{proof}
Из работы алгоритма 2 следует, что расписания $\pi(wL_{max}(\pi^{*}(N)))$ не существует, т.е. $\Phi(N,wL_{max}(\pi^{*}(N)) = \emptyset$. Cледовательно для любого $\pi' \in \Pi(N)$ выполнено
$$ \max\limits_{O_j\in N}w_j L_j(\pi') \geq \max\limits_{O_j\in N}w_j L_j(\pi^{*}(N)).$$
Первая часть теоремы доказана.

Если $$\pi^{*}(N) = \emptyset,$$ то не существует расписания $\pi(+\infty)$, и по теореме \ref{rzd:th:0} имеем $\Phi(N, +\infty) = \emptyset$.  Тогда, т.к. $\Phi(N, +\infty) = \Pi(N)$, то нет ни одного допустимого расписания, т.е. $\Pi(N) = \emptyset$. Теорема доказана.
\end{proof}

\begin{lemma}\label{rzd:lm:0,5}
Трудоёмкость алгоритма \ref{rzd:alg:0,5} $O(q n^2 \log n)$ операций, где $n$ --- количество заказов, $q$ --- количество составов.
\end{lemma}
\begin{proof}
 Без ограничения общности будем считать, что требование $O_j \in N$, на котором было достигнуто значение целевой функции $wL_{max}(\pi_i)$ при построении расписания $\pi_i$ было отправлено поездом $a$, т.е. $O_j \in B^a(\pi_i)$. Тогда при любом расписании $\pi(y)$ при значении $y < wL_{max}(\pi_i)$ для требования $O_j$  будет выполнено
$$T^{max}(O_j,y) < a.$$
Таким образом, при каждом проходе цикла 1-2 алгоритма \ref{rzd:alg:0,5} одно из требований должно быть отправлено поездом с меньшим номером, чем при расписании, построенном на предыдущем проходе цикла 1-2. Всего требований $n$, для каждого требования значение $T^{max}(O_j,y)$ при изменении значения $y$ может уменьшаться не более чем $q-1$ раз.
Следовательно, если $\pi^{*}(N) = \pi_l$, то $l \leq n(q-1)$. Следовательно алгоритм \ref{rzd:alg:0} решит вспомогательную задачу не более чем $n(q-1)$ раз. Трудоёмкость решения вспомогательной задачи с помощью алгоритма \ref{rzd:alg:0} составляет $O(n \log n)$ операций. Таким образом, общая трудоёмкость алгоритма \ref{rzd:alg:0,5} составляет $O(q n^2 log n)$ операций. Лемма доказана.
\end{proof}

Для задачи \ref{rzd:pr:1} минимизации максимального ВВС выполнения заказов
для двух станций с фиксированными моментами отправления поездов предложен алгоритм нахождения оптимального расписания трудоёмкостью $O(q n^2 log n)$ операций, где $q$ --- количество поездов, а $n$ --- количество заказов.

\subsection{Вспомогательная задача для задачи \ref{rzd:pr:2}}\label{rzd:sec:aux2:4}
Прежде чем сформулоировать вспомогательную задачу введём дополнительные обозначения.

Введем функцию $\lceil t \rceil$, определённую на множестве моментов времени $t$ для заданного множества допустимых интервалов отправления $U = \{[u^s_1, u^f_1), \dots, [u^s_v, u^s_v)\}$. Для любого момента времени $t$ значение функции вычисляется по формуле:
 $$\lceil t \rceil = \min\limits_{t' \in U} t' | t' \geq t.$$
Таким образом, $\lceil t \rceil \in U$ ---  это минимальный момент времени, не меньше $t$, принадлежащий ближайшему допустимому интервалу отправления.
При этом для любого $t > u^s_v$ значение функции $\lceil t \rceil$ будет равно $+\infty$.

Введем функцию $r(N')$, определенную на множестве заказов $N' \in N$ и заданную формулой:
 $$r(N') = \max\limits_{O_j \in N'}r_j,$$
 т.е. $r(U)$ --- момент времени, к которому поступят все заказы множества $U$.

Определим семейство множеств $N_0 \subseteq N_1 \subseteq \dots \subseteq N_q$ следующим образом. Пусть $N_0$ --- заказы, которые не могут быть отправлены ни одним из поездов, $N_1$ --- заказы, которые должны быть отправлены первым поездом, $N_2$ --- первым или вторым поездом и т.д. Для любого $i = 0,\dots, q$ мощность множества $N_i$ не может превышать количества заказов, которое могут быть отправлены первыми $i$ поездами, т.е.
\begin{equation}\label{rzd:eq:3}
|N_i| \leq K_i, i=0, \dots, q,
\end{equation}
где $K_i = \sum\limits_{j=1}^i k_j$ --- суммарное количество заказов (вагонов), перевозимых поездами $1,\dots, i$, а $K_0 = 0$. Очевидно имеем $N_q = N$.
Заметим, что определение множеств $N_0, \dots, N_q$, отличается от того, которое было сформулировано при решении задачи \ref{rzd:pr:1}. В данном случае, множества не зависят от моментов поступления $r_1, \dots, r_n,$ или других параметров заказов.

Для всех заказов $j=1, \dots, n$ обозначим через $N(O_j) = i$, $i=1, \dots, q$, номер множества $N_i,$ если для данного набора множеств $N_0, \dots, N_q$ заказ $O_j$ принадлежит множеству $N_i$ и не принадлежит $N_{i-1}$, т.е.
$$N(O_j) = \min\limits_{i=0, \dots, q} i | O_j \in N_i.$$

Будем говорить, что расписание $\pi$ \textit{удовлетворяет} набору множеств $N_0, \dots, N_q$, если для любого $i=1,\dots,q$ выполняется $N_i \subseteq B^1(\pi)\cup \dots \cup B^i(\pi)$, где $B^1(\pi), \dots, B^i(\pi)$ --- множества заказов, перевозимых поездами $1, \dots, i$ при расписании $\pi$.

Сформулируем вспомогательную задачу для задачи \ref{rzd:pr:2}.
Пусть задано множество заказов $N$ и ограничение на ВВС $y$. Необходимо составить допустимое расписание $\Theta(N,U,y) \in \Pi(N,U)$, удовлетворяющее критерию
\begin{equation*}
\min\limits_{\pi \in \Pi(N,U)}C_{max}(\pi) | \max\limits_{O_j\in N} w_j L_j(\pi) < y,
\end{equation*}
где
$$C_{max}(\pi) = \max\limits_{O_j \in N} C_j(\pi),$$
либо показать, что для данного значения $y$ не существует допустимого расписания. Для решения данной задачи применим подход, который впервые был применён для задачи $1|r_j|L_{max}$ в \cite{Laz:08}.

Введем дополнительные обозначения. Из ограничения на максимальное ВВС, для любого $O_j \in N$ имеем:
$$w_j(C_j(\pi) - d_j) < y,$$
следовательно,
$$C_j(\pi) < \frac{y}{w_j} + d_j.$$
Для заданного значения ограничения $y$  для каждого заказа $O_j \in N$ может быть определён \textit{дидлайн} --- момент времени, до которого данный заказ должен быть доставлен на станцию назначения:
$$D_j(y) = \frac{y}{w_j} + d_j.$$

Если значение времени доставки $C_j(\pi)$ не превышает $D_j(y)$, то заказ $O_j$ не нарушает ограничения на ВВС при расписании $\pi$. Расписание $\pi$ будем называть \textit{разрешённым} если для любого заказа $O_j \in N$, перевозимого поездом $B^i(\pi)$, момент поступления заказа $r_j$ не превышает момент отправления поезда $S^i(\pi)$ и момент прибытия поезда $C^i(\pi)$ не превышает дидлайна заказа $D_j(y),$ т.е. выполнены неравенства
$$S^i(\pi) \geq r_j,$$
$$C^i(\pi) < D_j(y).$$
Множество всех разрешённых расписаний для заданного ограничения $y$ обозначим через $\Phi(N,U,y) \subseteq \Pi(N,U)$.

Вспомогательная задача может быть сформулирована следующим образом: имеется множество заказов $N$ для каждого из которых определены момент поступления $r_j$ и дидлайн $D_j(y)$. Также имеется множество интервалов допустимого отправления $U$ и набор множеств $N_0, N_1, \dots, N_q$. Требуется найти расписание $\Theta(N,U,N_0, N_1, \dots, N_q,y) \in \Phi(N,U,y),$ для которого выполнено
\begin{equation}\label{rzd:eq:2}
C_{max}(\Theta(N,U,N_0, N_1, \dots, N_q,y)) = \min\limits_{\pi \in \Phi(N,U,y)} C_{max}(\pi),
\end{equation}
при условии, что $\pi$ удовлетворяет множествам $N_0, N_1, \dots, N_q$.

Рассмотрим дополнения $\overline{N_i} = N \setminus N_i$ при $i =1, \dots, q$. Из определения множества $N_i$ следует, что любой поезд с номером больше чем $i$ должен состоять только из заказов, принадлежащих множеству $\overline{N_i}$, т.е. для любого $l = i+1, \dots, q$ будет выполнено $ B^l \subseteq \overline{N_i}$.
Докажем следующую лемму об отправлении поездов.

\begin{lemma}\label{rzd:lm:1}
При любом расписании $\pi \in \Phi(N,U,y)$, удовлетворяющем множествам $N_1, \dots, N_q,$ для любого значения $i =1, \dots, q$ момент отправления поезда $S^i$ и множество заказов поезда $B^i$ должны удовлетворять следующим свойствам:
  \begin{enumerate}
  \item к моменту времени $r(B^i)$ поступления заказов из множества $B^i$ должны поступить хотя бы $K_i = \sum\limits_{j=1}^i k_j$ заказов, в том числе все заказы из $N_i$, т.е. \label{rzd:prop:rTm}
  $$r(B^i) \geq \max\{r(N_i), r_{K_i}\};$$
  \item момент времени $S^i$ отправления поезда $i$ удовлетворяет следующему неравенству\label{rzd:prop:ReleaseTrain}
  \begin{equation} \label{rzd:eq:4}
    S^i \geq \lceil \max \{r(B^i), S^{i-1} + \widehat{u}\} \rceil.
  \end{equation}
Будем предполагать, что $S^0 = -\infty.$
\end{enumerate}
\end{lemma}

\begin{proof}
\begin{enumerate}
\item Свойство \ref{rzd:prop:rTm} следует из того, что для отправления первых $i$ поездов, $i=1, 2, \dots, q,$ необходимо, чтобы к моменту времени $S^i$ поступило хотя бы $K_i = |B^1| + |B^2| +\dots + |B^i|$ заказов. Так как расписание $\pi$ удовлетворяет набору множеств $N_1, \dots, N_q$, то к моменту $S^i$ отправления поезда $i$ должны поступить все заказы из множества $N_i$.
\item Свойство \ref{rzd:prop:ReleaseTrain}. Моменты отправления поездов должны быть разделены интервалом времени, не меньшим чем $\widehat{u}$. Следовательно
$$S^i \geq \max\{r(B^i), S^{i-1} + \widehat{u}\}.$$
Кроме того, момент отправления поезда $S^i$ должен принадлежать одному из допустимых интервалов, т.е.
$$S^i \geq \lceil \max\{r(B^i), S^{i-1} + \widehat{u}\} \rceil.$$
и неравенство (\ref{rzd:eq:4}) выполняется.
\end{enumerate}
Лемма доказана.
\end{proof}


\subsection{Алгоритм решения вспомогательной задачи}\label{rzd:sec:aux2c:6}

Алгоритм построения расписания $\Theta(N,U,N_0, N_1, \dots, N_q, y)$, оптимального по критерию $C_{max}$ для заданного множества заказов $N$, множества допустимых интервалов отправки $U$, набора множеств $N_0, N_1, \dots, N_q$ и ограничения на ВВС $y$ заключается в следующем.

Для всех заказов $O_j \in N $ для заданного значения ограничения $y$ вычисляются значения дидлайнов $D_j(y),$ выполняются присвоения $N_0^0 = N_0, \dots, N_q^0 = N_q$, после чего алгоритм выполняется итерационно.
На $m$-ой итерации алгоритма последовательно выпоняются следующие шаги.
На первом шаге рассматривается формирование последнего поезда $q$ (рисунок \ref{rzd:pic:2}). С учётом того, что $B^q \subseteq \overline{N^m_{q-1}}$ и $|B^q| = k_q$, выбираем из множества $\overline{N^m_{q-1}}$ необходимое количество заказов $k_q$ с наибольшими значениями моментов поступления $r_j$ и включаем их в поезд $q$. После чего переходим к формированию поезда $q-1$.


\begin{figure}[h!]
\center{\includegraphics[width=16cm]{rzd_new_3.png}}
\caption{Формирование поезда $q$.}
\label{rzd:pic:2}
\end{figure}

\begin{figure}[h!]
\center{\includegraphics[width=16cm]{rzd_new_4.png}}
\caption{Формирование поезда $q-1$.}
\label{rzd:pic:3}
\end{figure}

Так как один заказ не может быть включен в два разных поезда, то поезд $q-1$ может перевезти только заказы из $\overline{N^m_{q-2}}$, не включенные в $B^q$, т.е. $B^{q-1} \subseteq (\overline{N^m_{q-2}} \setminus B^q)$ (рисунок \ref{rzd:pic:3}). Выбираем $k_{q-1}$ заказов из множества $\overline{N^m_{q-2}} \setminus B^q$ и включаем их в $B^{q-1}$. На каждом следующем шаге поступаем аналогично. Таким образом при формировании поезда $i$ выбираем $k_i$ заказов с наибольшими моментами поступления из множества $\overline{N^m_{i-1}} \setminus (B^{i+1} \cup B^{i+2} \cup \dots \cup B^q)$ и включаем их в $B^i$. Так как для любого $l=1, \dots, q$ выполнено условие (\ref{rzd:eq:3}),
%$$|N_i| \leq \sum\limits_{j=1}^i k_j,$$
то для любого $l=2, \dots, q$ будет выполнено неравенство
$$ |\overline{N^m_{l-1}}| \geq \sum\limits_{j=l}^q k_j.$$

После того, как все поезда сформированы, вычисляем моменты отправления $S^i$ поездов $i=1, \dots, q$ по формуле
$$S^i = \lceil \max \{r(B^i), S^{i-1} + \widehat{u}\} \rceil.$$
Последовательно находим значения
$S^1 = \lceil r(B^1) \rceil,$ $S^2 = \lceil \max\{r(B^2), S^1 + \widehat{u}\} \rceil$, $\dots$, $S^q = \lceil \max\{r(B^q), S^{q-1} + \widehat{u}\} \rceil$.

\begin{figure}[h!]
\center{\includegraphics[width=16cm]{rzd_new_2.png}}
\caption{Включение заказа $O_j$ в $N^m_{i-1}$.}
\label{rzd:pic:4}
\end{figure}

Для всех заказов $O_j$, $j=1, \dots, n$ поочерёдно пересчитываем значения $N(O_j)$, последовательно проверяя неравенства
$$C^i(\pi) \geq D_j(y),$$
для $i=N(O_j), \dots, 1$. Если данное неравенство верно для $i=l$, где $O_j \in B^l(\pi)$, то построенное расписание не удовлетворяет ограничению $wL_{max} (\pi) < y.$
Если неравенство выполняется, то заказ $O_j$ включается во множество $N^m_{i-1}$, выполняется присвоение значения $N(O_j) = i-1.$ Если же неравенство не выполняется, то т.к. $C^i(\pi) \geq C^{i-1}(\pi) \geq \dots \geq C^1(\pi)$, дальнейшая проверка не имеет смысла.

После того, как проверка выполнена для всех $O_j$, $j=1, \dots, n$, и новые множества $N^m_1, N^m_2, \dots N^m_q$ составлены, проверяем их мощность. В случае, если для какого-либо множества $i$, $i=1, \dots, q$, верно неравенство
$$|N^m_i| > K_i,$$
 то алгоритм прерывается. Таким образом нарушается условие (\ref{rzd:eq:3}), следовательно, разрешённого расписания $\Theta(N,U,N_0, N_1, \dots, N_q, y)$ не существует. Если мощности всех множеств удовлетворяют данным ограничениям, выполняются присвоения $m:=m+1$ и $N^m_i := N_i^{m-1}$ для всех $i=1, \dots, q,$ после чего происходит переход на следующую итерацию. Выполнение алгоритма продолжается до тех пор, пока не будет построено расписание $\pi$ такое, что для любого $O_j \in B^l(\pi)$, $j = 1,\dots, n$, выполнено
 $$C^l(\pi) < D_j(y),$$
или не будет нарушено условие (\ref{rzd:eq:3}).

 %Формальная запись алгоритма выглядит следующим образом:
 \begin{algorithm}[H]\label{rzd:alg:1}
 \NoCaptionOfAlgo
\caption{\textbf{Алгоритм \ref{rzd:alg:1}}}
\begin{itemize}
\item[0.] Входные данные:\\
 $N, U, y, q, m=0, k_1, \dots, k_q, N_0, N_1, \dots, N_{q-1}; N_q = N.$
\item[1.] Для всех заказов $j = 1,\dots, n$ вычисляем значения $D_j(y) := \frac{y}{w_j} + d_j,$ $N(O_j) = \min\limits_{i=1, \dots, q} i | O_j \in N_i.$ Для всех $i=0,1, \dots, q$ присваиваем $N_i^m := N_i$.
\item[2.] Для значений $i=q, \dots, 1$ последовательно формируем составы:\\
 в состав $B^i(\pi^m)$ включаем $k_i$ заказов из множества $\overline{N^m_{i-1}} \setminus (B^{i+1}(\pi^m) \cup \dots \cup B^q(\pi^m))$ с наибольшими значениями моментов поступления и вычисляем значения $r(B^i(\pi))$.
\item[3.] Для значений $i=1, \dots, q$ вычисляем моменты отправления поездов по формуле:
$$S^i(\pi^m) = \lceil \max \{r(B^i(\pi^m)), S^{i-1}(\pi^m) + \widehat{u}\} \rceil,$$
учитывая $S^0(\pi^m) = -\infty.$
\item[4.] Для заказов с номерами $j=1, \dots, n$ и поездов $i=N(O_j), \dots, 1$, последовательно проверяем выполнимость неравенства
$$C^i(\pi^m) < D_j(y).$$
 \begin{itemize}
    \item[a)] Если для какой-то пары $j$ и $i$, неравенство не верно, выполняется включение заказа $O_j$ во множество $N^m_{i-1}$, изменение значения $N(O_j) := i-1$ после чего проверка продолжается.
    \item[b)] Если для всех пар $j, i$, где $j=1, \dots, n$, $O_j \in B^i(\pi)$ неравенство верно, то
    $$\Theta(N,U,N_0, N_1, \dots, N_q, y)) = \pi^m$$
     и алгоритм завершает работу.
 \end{itemize}
\item[5.] Последовательно проверяем выполнимость условия (\ref{rzd:eq:3})
$$|N^m_i| \leq K_i$$
 для значений $i=0,1, \dots, q.$
    \begin{itemize}
    \item[a)] Если для всех множеств условие (\ref{rzd:eq:3}) выполнено, присваиваем $m:=m+1,$ для значений $i=0,1, \dots, q$, присваиваем $N_i^m := N_i^{m-1}$ и переходим на шаг 2.
    \item[b)] Если для какого-то множества условие (\ref{rzd:eq:3}) не выполнено, алгоритм завершает работу и $\Theta(N,U,N_0, N_1, \dots, N_q, y) = \emptyset$.
    \end{itemize}
\end{itemize}
\end{algorithm}

\normalsize
\begin{lemma}\label{rzd:lm:2}
Пусть $\pi$ --- расписание, построенное на некоторой итерации алгоритма \ref{rzd:alg:1} для набора множеств $N_0, N_1, \dots, N_q$. Тогда для любого расписания $\pi' \in \Phi(N,U,y)$, удовлетворяющего множествам $N_0, N_1, \dots, N_q,$ для любого поезда $i = 1, \dots, q$ выполняется неравенство:
$$S^i(\pi) \leq S^i(\pi'),$$
т.е. при расписании $\pi$ каждый из $q$ поездов отправляется в минимально возможное время.
\end{lemma}

\begin{proof}
Будем сравнивать составы поездов расписаний $\pi$ и $\pi'$, перебирая их в обратном порядке. Пусть на каком-то шаге, среди множества заказов $B^l(\pi)$, перевозимых поездом $l$, нашелся заказ $O_x$, который отсутствует в $B^l(\pi')$ (рисунок \ref{rzd:pic:5}). Так как $k_l = |B^l(\pi)| = |B^l(\pi')|$, в составе $B^l(\pi')$ должен присутствовать заказ $O_z$, который отсутствует в $B^l(\pi)$.

\begin{figure}[h!]
\center{\includegraphics[width=16cm]{rzd_new_5.png}}
\caption{Отличие в составах $B^l(\pi)$ и $B^l(\pi')$.}
\label{rzd:pic:5}
\end{figure}

Заметим, что $B^l(\pi) \neq B^l(\pi')$ --- первое из встретившихся отличий, т.е. $B^{l+1}(\pi) = B^{l+1}(\pi'), \dots, B^{q}(\pi) = B^{q}(\pi')$. Из того, что заказ $O_x \in B^l(\pi)$, но отсутствует во множествах $B^l(\pi'), B^{l+1}(\pi'), \dots, B^q(\pi')$ следует, что при расписании $\pi'$ он включен в состав $B^a(\pi')$, где $a < l$. Аналогичные рассуждения проводим для заказа $O_z$, делая вывод, что при расписании $\pi$ $O_z$ включён в состав $B^b(\pi)$, где $b < l$ (рисунок \ref{rzd:pic:5}).

Так как при формировании состава $B^l(\pi)$ алгоритмом \ref{rzd:alg:1} были выбраны заказы с наибольшими значениями $r_j$, можно сделать вывод, что $r_{x} \geq r_{z}$, и т.к. $O_x \in B^a(\pi')$, то
$$S^a(\pi') \geq r_{x} \geq r_{z}.$$

Кроме того, $O_x, O_z \in \overline{N_{l-1}}$, т.к. оба расписания удовлетворяют набору множеств $N_1, N_2, \dots, N_q$. Отсюда следует, что при расписании $\pi'$ можно поменять заказы $O_x$ и $O_z$ местами (поездами) без сдвига моментов отправления поездов $S^1(\pi'), S^2(\pi'), \dots, S^q(\pi')$, и расписание будет удовлетворять набору множеств $N_1, N_2, \dots, N_q$ (рисунок \ref{rzd:pic:6}).

\begin{figure}[h!]
\center{\includegraphics[width=16cm]{rzd_new_6.png}}
\caption{Замена при расписании $\pi'$.}
\label{rzd:pic:6}
\end{figure}

Последовательно выполняя аналогичные операции, получаем расписание $\pi''$, удовлетворяющее множествам $N_1, N_2, \dots, N_q$, с составами $B^1(\pi), B^2(\pi), \dots, B^q(\pi)$ и моментами отправления $S^1(\pi'), S^2(\pi'), \dots, S^q(\pi')$. Из леммы \ref{rzd:lm:1} следует, что для любого $ i=1,\dots, q$ выполняется
$$S^i(\pi'') \geq \lceil \max(r(B^i), S^{i-1} + \widehat{u}) \rceil = S^i(\pi),$$
т.е. для любого $i=1, \dots, q$
$$S^i(\pi') = S^i(\pi'') \geq S^i(\pi).$$
Лемма доказана.
\end{proof}

\begin{lemma}\label{rzd:lm:2,5}
Пусть на некоторой итерации алгоритма \ref{rzd:alg:1} для набора множеств $N_0, N_1, \dots, N_q$ было построено расписание $\pi$ и в результате изменений множеств на шаге 4 для данного значения $y$ были получены множества $N'_0, N'_1, \dots, N'_q$ Тогда любе расписание $\pi' \in \Phi(N,U,y)$, удовлетворяющее множествам $N_0, N_1, \dots, N_q,$ удовлетворяет множествам $N'_0, N'_1, \dots, N'_q.$
\end{lemma}
\begin{proof}
По правилу включения на шаге 4 алгоритма $\ref{rzd:alg:1}$ для любого заказа $O_j$, и множеств $N_i$, $N'_i$, $i=0, 1, \dots, q$ таких что $O_j \in N_i$ и $O_j \notin N'_i$ должно быть выполнено
$$C^{i+1}(\pi) \geq D_j(y).$$
Расписания $\pi$ и $\pi'$ удовлетворяют множествам $N_0, N_1, \dots, N_q,$ тогда по лемме \ref{rzd:lm:2}, для любого $i=0, \dots, q-1$ выполнено
$$S^{i+1}(\pi) \leq S^{i+1}(\pi').$$
Так как для любого $i=0, \dots, q-1,$ $f_{i+1}(t)$ --- монотонно неубывающая функция, то
$$f_{i+1}(S^{i+1}(\pi')) \geq f_{i+1}(S^{i+1}(\pi)),$$
$$C^{i+1}(\pi') \geq C^{i+1}(\pi) \geq D_j(y).$$
Следовательно при любом расписании $\pi' \in \Phi(N,U,y)$, удовлетворяющем множествам $N_0, N_1, \dots, N_q,$ требование $O_j$ не может быть отправлено поездом с номером больше $i$. Аналогично рассматривая все заказы множеств $N'_0, N'_1, \dots, N'_q$ получаем, что расписание $\pi'$ удовлетворяет множествам $N'_0, N'_1, \dots, N'_q$. Лемма доказана.
\end{proof}

\begin{corollary}\label{rzd:col:1}
Пусть в результате построения расписания $\Theta(N,U,N_0, N_1, \dots, N_q, y)$ алгоритмом \ref{rzd:alg:1} были получены множества $N'_0, N'_1, \dots, N'_q$. Тогда любое расписание $\pi' \in \Phi(N,U,y),$ удовлетворяющее множествам $N_0, N_1, \dots, N_q,$ удовлетворяет множествам $N'_0, N'_1, \dots, N'_q.$
\end{corollary}
\begin{proof}
Данное утверждение следует из последовательного применения леммы \ref{rzd:lm:2,5} к каждой итерации алгоритма \ref{rzd:alg:1}.
\end{proof}

\begin{theorem}\label{rzd:th:1}
 Алгоритм \ref{rzd:alg:1} строит расписание $\Theta(N,U,N_0, N_1, \dots, N_q, y) \in \Phi(N,U,y)$, имеющее минимальной время обслуживания заказов $C_{max}$ среди всех расписаний из $\Phi(N,U,y)$, удовлетворяющих $N_0, N_1, \dots, N_q$.\\
%$$C_{max}(\Theta(N,U,N_0, N_1, \dots, N_q, y)) = \min\limits_{\pi \in \Phi(N,U,y)} C_{max}(\pi).$$
%\begin{equation*}
%\min\limits_{\pi \in \Phi(N,y)}C_{max}(\pi).
%\end{equation*}
Если $\Theta(N,U,N_0, N_1, \dots, N_q, y) = \emptyset$, то не существует расписания $\pi^* \in \Phi(N,U,y)$, удовлетворяющего множествам $N_0, N_1, \dots, N_q$.
 %$$\Phi(N,U,y) = \emptyset.$$
\end{theorem}

\begin{proof}
По следствию \ref{rzd:col:1} из леммы \ref{rzd:lm:2,5} любое расписание $\pi' \in \Phi(N,U,y),$ удовлетворяющее множествам $N_0, N_1, \dots, N_q,$ будет удовлетворять множествам $N'_0, N'_1, \dots, N'_q,$ построенным на последней итерации алгоритма \ref{rzd:alg:1}. По лемме \ref{rzd:lm:2} для любого расписания $\pi' \in \Phi(N,U,y),$ удовлетворяющего множествам $N'_0, N'_1, \dots, N'_q,$ для любого $i=1, \dots, q$ будет выполнено
$$S^i(\Theta(N,U,N_0, N_1, \dots, N_q, y)) \leq S^i(\pi').$$
Тогда выполняется неравенство
$$S^{q}(\Theta(N,U,N_0, N_1, \dots, N_q, y)) \leq S^{q}(\pi').$$
Так как $f_{q}(t)$ --- монотонно неубывающая функция, то
$$f_{q}(S^{q}(\Theta(N,U,N_0, N_1, \dots, N_q, y))) \geq f_{q}(S^{q}(\pi')),$$
$$C^{q}(\Theta(N,U,N_0, N_1, \dots, N_q, y)) \leq C^{q}(\pi').$$
Для любого расписания $\pi$
$$C_{max}(\pi) = C^{q}(\pi'),$$
следовательно
$$C_{max}(\Theta(N,U,N_0, N_1, \dots, N_q, y)) \leq C_{max}(\pi').$$
Первое утверждение теоремы доказано.

Докажем второе утверждение Пусть алгоритм завершился на $m$-ой итерации и при этом для некоторого $i = 0, 1, \dots, q$ не было выполнено условие (\ref{rzd:eq:3})
$$|N^m_i| > K_i.$$
Тогда по следствию \ref{rzd:col:1} из леммы \ref{rzd:lm:2,5} любое расписание $\pi' \in \Phi(N,U,y),$ удовлетворяющее множествам $N_0, N_1, \dots, N_q,$ удовлетворяет множеству $N^m_i,$ а значит нарушает условие (\ref{rzd:eq:3}). Противоречие.

Теорема доказана.
\end{proof}

\begin{corollary}\label{rzd:col:1,5}
Если $\Phi(N,U,y) \neq \emptyset$, то
$$C_{max}(\Theta(N,U,\emptyset, \dots,\emptyset, N,y)) = \min\limits_{\pi \in \Phi(N,U,y)} C_{max}(\pi).$$
\end{corollary}

\subsection{Алгоритм решения задачи минимизации максимального ВВС выполнения заказов для двух станций}\label{rzd:sec:main2:7}
Будем обозначать ВВС заказа $O_j$ при расписании $\pi$ через $w_j L_j(\pi)$. Максимальное ВВС заказа при расписании $\pi$ обозначим через $wL_{max}(\pi)$.
При построении оптимального расписания $\pi^{*}(N,U)$ будем действовать следующим образом.

На первом шаге построим расписание $\pi_1 = \Theta(N,U,\emptyset, \dots,\emptyset, N, +\infty)$, при котором заказы отправляются по возрастанию моментов поступления, и включим его во множество $\Omega(N, U)$. Тем самым будет построено расписание, удовлетворяющее условию минимума $C^i$ для каждого поезда $i$, $i=1, \dots, q$, а значит и критерию $C_{max}$, а в результате работы алгоритма \ref{rzd:alg:1} будут построены множества $N_0^1, N_1^1, \dots, N_q^1.$

Рассмотрим заказ $O_{j_1}$ на котором достигается максимум функции ВВС $w_{j_1} L_{j_1}$ при расписании $\pi_1$. Пусть этот заказ при данном расписании отправляется $m_1$-ым поездом, т.е. $O_{j_1} \in B^{m_1} (\pi_1)$. По лемме \ref{rzd:lm:1} в результате работы алгоритма \ref{rzd:alg:1} построено расписание с минимальными моментами отправления поездов из всех возможных, удовлетворяющих ограничению $wL_{\max}<y$. Таким образом, для любого расписания $\pi'$, удовлетворяющего
$$w_{j_1}L_{j_1} \leq wL_{\max}(\pi')<y,$$
будет выполнено
$$C^i(\pi') \geq C^i(\pi_1).$$
Следовательно при любом расписании $\pi'$, удовлетворяющем
$$wL_{\max} (\pi') < wL_{\max} (\pi_1),$$
заказ $j_1$ должен прибыть на станцию назначения до момента времени $C^i(\pi')$.
Это значит, что если множество $\Phi(N,U, w_{j_1} L_{j_1})$ не пустое, то при любом расписании из множества $\Phi(N,U, w_{j_1} L_{j_1})$ заказ $O_j$ будет отправлен одним из поездов, номер которого меньше $m_1$.

Построим расписание $\Theta(N,U,N_0^1, N_1^1, \dots, N_q^1, w_{j_1} L_{j_1})$, при котором максимальное ВВС будет достигнуто на заказе $O_{j_2}$, отправленным $m_2$-ым поездом.  Включим данное расписание во множество $\Omega(N,U)$. В результате построения расписания $\Theta(N,U,N_0^1, N_1^1, \dots, N_q^1, w_{j_1} L_{j_1})$ с помощью алгоритма \ref{rzd:alg:1} будут построены множества $N_0^2, N_1^2, \dots, N_q^2$.

Следующим шагом будет выполняться построение расписания $\Theta(N,U,N_0^2, N_1^2, \dots, N_q^2 w_{j_2} L_{j_2})$, при котором, заказ $O_{j_2}$ должен быть отправлен поездом с номером меньше, чем $m_2$, а значит заказ $O_{j_2}$ попадет во множество $N_{m_2 - 1}$. Данное расписание также включим в $\Omega(N,U)$.

Будем повторять эту процедуру до тех пор пока не наступит такой шаг $s+1$, что расписания $\Theta(N,U,\Theta(N,U,N_0^{s}, N_1^{s}, \dots, N_q^{s}, w_{j_1} L_{j_1}), w_{j_s} L_{j_s})$ не существует. В этом случае, алгоритм \ref{rzd:alg:2} возвращает множество $\Omega(N,U)$ и расписание $\pi^{*}(N,U) = \pi_s$ и завершает свою работу.

%Формальная запись алгоритма \ref{rzd:alg:2} выглядит следующим образом:
\begin{algorithm}[H]\label{rzd:alg:2}
\NoCaptionOfAlgo
\caption{\textbf{Алгоритм \ref{rzd:alg:2}}}
\begin{itemize}
\item[0.] Входные данные:\\
$N, U, k_1, \dots, k_q, y = +\infty, s=0.$
\item[1.] Для всех $i=1, \dots, q-1$ выполняется присвоение: $N^s_i := \emptyset.$ Также присваиваем $N^s_q := N$.
\item[2.] Выполняется построение расписания $\pi_{s+1} = \Theta(N,U,N_0^s, N_1^s, \dots, N_q^s,y)$ с помощью алгоритма \ref{rzd:alg:1}. В результате работы алгоритма \ref{rzd:alg:1} получены множества $N_0^{s+1}, N_1^{s+1}, \dots, N_q^{s+1}.$
\item[3.] Если $\pi_{s+1} \neq \emptyset:$
    \begin{itemize}
    \item[a)] включаем $\pi_{s+1}$ в $\Omega(N,U)$;
    \item[b)] присваиваем $\pi^{*}(N,U) := \pi_{s+1}$;
    \item[c)] присваиваем $y := wL_{max}(\pi_{s+1})$, $s:=s+1$ и переходим на шаг 2.
    \end{itemize}
\item[4.] Если $\pi_{s+1} = \emptyset$, алгоритм возвращает множество $\Omega(N,U)$, расписание $\pi^{*}(N,U)$ и завершает работу.
\end{itemize}
\end{algorithm}

\begin{theorem}\label{rzd:th:2}
Множество расписаний $\Omega(N,U)$, полученное в результате выполнения алгоритма \ref{rzd:alg:2}, является оптимальным по Парето для критериев $wL_{max}$ и $C_{max}$.
Расписание $\pi^{*}(N,U)$, полученное в результате выполнения алгоритма \ref{rzd:alg:2}, оптимально по критерию
$$ \min\limits_{\pi \in \Pi(N,U)} \max\limits_{O_j\in N} C_j(\pi) | \min\limits_{\pi \in \Pi(N,U)}\max\limits_{O_j\in N} w_j L_j(\pi).$$
\end{theorem}

\begin{proof}
Заметим, что $\Pi(N,U) = \Phi(N,U, +\infty)$. Тогда по следствию \ref{rzd:col:1,5} из теоремы \ref{rzd:th:1} получаем, что расписание $\Theta(N,U,\emptyset, \dots,\emptyset, N, +\infty)$ является оптимальным по критерию $C_{max}$. На каждом проходе цикла 2-3 алгоритма \ref{rzd:alg:2} выполняется построение расписания по алгоритму \ref{rzd:alg:1}. По теореме \ref{rzd:th:1} имеем:
$$C_{max}(\pi_1) < C_{max}(\pi_2)) < \dots < C_{max}(\pi_s),$$
$$wL_{max}(\pi_1) > wL_{max}(\pi_2) > \dots > wL_{max}(\pi_s).$$
Таким образом, если будет доказано, что расписание $\pi^{*}(N,U)$ является оптимальным по критерию $wL_{max}$, то из этого будет следовать, что множество расписаний $\Omega(N,U)$ оптимально по Парето.
Пусть расписание $\pi^{*}(N,U)$ имеет значение целевой функции равное $y^{*}$. Из несуществования расписания $\Theta(N,U,N_0^s, N_1^s, \dots, N_q^s, y^{*})$ по теореме \ref{rzd:th:1} следует, что расписаний, удовлетворяющих построенным множествам $N_0^s, N_1^s, \dots, N_q^s$ и неравенству $(wL)_{\max}<y^{*}$, не существует. Последовательно применяя следствие \ref{rzd:col:1} из леммы \ref{rzd:lm:2,5} к построению расписаний $\pi_1, \dots, \pi_s$ получаем, что не существует расписания $\pi \in \Phi(N,U,y)$, удовлетворяющего множествам $N_0 = \emptyset, N_1 = \emptyset, \dots, N_{q-1} = \emptyset, N_q = N,$ т.е $\Phi(N,U,y) = \emptyset$.
Получаем, что не существует расписания $\pi \in \Pi(N,U),$
 для которого выполнено неравенство
 $$\max\limits_{O_j \in N} w_j L_j(\pi) < y^{*}.$$
 Следовательно, расписание $\pi^{*}(N,U)$ оптимально по критерию $wL_{\max}$. Из теоремы \ref{rzd:th:1} также следует, что расписание $\pi^{*}(N,U)$ имеет наименьшее общее время доставки заказов.
 Теорема доказана.
 \end{proof}

\subsection{Оценка трудоёмкости алгоритма}\label{rzd:sec:comp:8}
\begin{theorem}\label{rzd:th:3}
Трудоёмкость алгоритма \ref{rzd:alg:2} составляет $O(n^2 \max\{n \log n, q \log v\})$ операций, где $n$ ---количество заказов.
\end{theorem}
\begin{proof}
Для начала рассмотрим трудоёмкость алгоритма \ref{rzd:alg:1}. На шаге 2 выполняется подбор $k_q, \dots, k_1$ заказов для формирования составов $B^q, \dots, B^1$ и вычисление значений $r(B^q(\pi)), \dots, r(B^1(\pi))$. Следовательно, трудоёмкость шага 2, с учётом сортировки заказов по моментам поступления, составляет $O(n \log n)$ операций. Вычисление моментов отправления на шаге 3 для $q$ поездов имеет трудоёмкость $O(q log v)$ операций, где $v$ --- количество допустимых интервалов отправки во множестве $U$. Трудоёмкость проверки $O(n)$ неравенств на шаге 5 составляет $O(n)$ операций. Вклад в трудоёмкость проверки неравенств на шаге 4 будет оценена позже.
Таким образом, одна итерация алгоритма \ref{rzd:alg:1} (без учёта шага 4) имеет трудоёмкость $O(\max\{n \log n, q \log v\})$ операций.

На каждой итерации алгоритма \ref{rzd:alg:1} происходит добавление заказа в одно или несколько множеств $N_1, N_2, \dots , N_{q-1}$. Так как алгоритм \ref{rzd:alg:1} строит расписание с минимальными моментами отправления из всех, удовлетворяющих ограничению $wL_{\max}<y$, то при уменьшении значения $y$ для существования расписания необходимо, чтобы заказ, на котором была достигнуто максимальное значение ВВС был отправлен поездом с меньшим порядковым номером. Следовательно, на каждом проходе цикла шагов 2-3 алгоритма \ref{rzd:alg:2} происходит включение заказа, на котором достигнута целевая функция, в одно из множеств $N_1, N_2, \dots , N_{q-1}$. Следовательно, суммарное число итераций алгоритма \ref{rzd:alg:1} при проходе цикла 2-3 алгоритма \ref{rzd:alg:2} не превышает суммы мощностей множеств $N_1, N_2, \dots , N_{q-1}$,
$$\sum\limits_{i=1}^{q} |N_i| \leq \sum\limits_{i=1}^{q-1} (q-i)k_i < n^2.$$
Из приведенных рассуждений следует, что общая трудоёмкость выполнения шагов 1, 2, 3 и 5 алгоритма \ref{rzd:alg:1} при выполнении алгоритма \ref{rzd:alg:2} составляет $O(n^2 \max\{n \log n, q \log v\})$ операций.

Теперь рассмотрим вклад выполнения шага 4 алгоритма \ref{rzd:alg:1} в трудоёмкость алгоритма \ref{rzd:alg:2}. Заметим, что для каждого требования $O_j$, $j=1, \dots, n,$ количество выполнения шагов 4a) алгоритма \ref{rzd:alg:1} равно $n-N(O_j)$, где $N(O_j)$ --- наименьший индекс множества $N_1, \dots, N_n$, которому принадлежит заказ $O_j$ после завершения работы алгоритма \ref{rzd:alg:2}. Таким образом количество шагов 4a) алгоритма \ref{rzd:alg:1} не превышает суммы мощностей множеств $N_1, N_2, \dots , N_{q-1}$
$$\sum\limits_{i=1}^{q} |N_i| \leq \sum\limits_{i=1}^{q-1} (q-i)k_i < n^2.$$
Количество обращений к шагу 4a) равна $O(1)$, следовательно общая трудоёмкость шагов 4a) алгоритма \ref{rzd:alg:1} при выполнении алгоритма \ref{rzd:alg:2} составляет $O(n^2)$.
Количество шагов 4b) алгоритма \ref{rzd:alg:1} равно количеству расписаний во множестве $\Omega(N,U)$, что не превышает количество итераций алгоритма \ref{rzd:alg:1} при проходе цикла 2-3 алгоритма \ref{rzd:alg:2}. Трудоёмкость шага 4b) равна $O(1)$. Таким образом вклад шага 4b) \ref{rzd:alg:1} в трудоёмкость алгоритма \ref{rzd:alg:2} составляет $O(n^2)$ операций.

Таким образом трудоёмкость алгоритма \ref{rzd:alg:2} составляет $O(n^2 \max\{n \log n, q \log v\}$ операций. Теорема доказана.
\end{proof}
\begin{corollary}\label{rzd:col:2}
Мощность множества $\Omega(N,U)$ не превышает $\sum\limits_{i=1}^{q-1} (q-i)k_i$.
\end{corollary}

\subsection{Заключение}\label{rzd:sec:concl:9}
  В статье рассмотрены две постановки задачи минимизации максимального взвешенного временн\'{о}го смещения доставки заказов между двумя станциями на железнодорожном транспорте. Для постановки с фиксированными моментами отправления и прибытия поездов предложен алгоритм трудоёмкости $O(q n^2 log n)$ операций.

  Для постановки с неопределёнными моментами отправления и прибытия поездов была рассмотрена бикритериальная задача минимизации максимального взвешенного временн\'{о}го смещения и общего времени доставки заказов между двумя железнодорожными станциями. Представлен алгоритм трудоёмкости $O(n^2 \max\{n \log n, q \log v\})$ операций, позволяющий построить множество Парето-оптимальных расписаний. Общее количество расписаний, входящих в Парето-множество не превышает
$$\sum\limits_{i=1}^{q-1} (q-i)k_i,$$
что в свою очередь меньше чем $n^2$. Доказано, что предложенный алгоритм не только минимизирует общее время доставки заказов, но и время отправления каждого из поездов. 