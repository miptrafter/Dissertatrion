\chapter{Результаты теоретических исследований} \label{chapt1}
\section{Новый полиномиально разрешимый случай для задачи минимизации максимального временн\'{о}го смещения} \label{Lmax_ab:sec:1}
\subsection{Введение}
Рассматривается следующая задача теории расписаний. Начиная с момента времени $t$ необходимо обслужить требования (работы) множества $N = \{1, \dots, n\}.$
Запрещаются прерывания обслуживания требований и одновременное обслуживание нескольких требований.

Для требований множества $N$ введем следующие обозначения: $r_j$ -- \textit{момент поступления} требования $j$, $p_j > 0$ -- время, которое требуется для обслуживания требования $j$, $d_j$ -- директивный срок, $j \in N$. \textit{Директивный срок} -- время, до которого желательно (но необязательно) завершить обслуживание требования. Обозначим: $r_j(t) = \max\{r_j, t\}$, $j \in N$. \textit{Расписанием} $\pi(N,t)$ будем называть последовательность обслуживания требований множества $N$ $$\pi(N,t) = (O_1, \dots, O_n),$$ начиная с момента времени $t$, где $O_1 \cup \dots \cup O_n \equiv N$ и обслуживание требования $O_1$ начинается в момент времени $s_1 = r_{O_1}(t)$, а всех остальных требований $O_j$ ($j=2, \dots, n$) начинается в момент времени $s_{O_j} = r_{O_j}(s_{O_{j-1}} + p_{O_{j-1}})$. Будем называть расписание \textit{допустимым}, если при нем каждое требование  $j \in N$ обслуживается без прерываний на протяжении времени $p_j$, начиная с момента $s_j \geq r_j(t)$, и никакие два требования не обслуживаются одновременно. Множество всех допустимых расписаний, которые можно построить для набора требований $N$ и момента времени $t$, обозначим через $\Pi(N, t)$.
Для каждого требования $j \in N$ в расписании $\pi \in \Pi(N,t)$ через $C_j(\pi, t)$ будем обозначать время завершения обслуживания $j$. Разность $L_j(\pi, t) = C_j(\pi, t) - d_j$ называют \textit{временн\'{ы}м смещением} требования $j$ при расписании $\pi$, начинающемся в момент времени $t$ (рисунок \ref{Lmax_ab:ris:1}). Максимальное временн\'{о}е смещение для требований множества $N$ при расписании $\pi$ определим как
$$L_{\max}(\pi,t) = \max\limits_{j \in N} C_j(\pi,t) - d_j.$$
Момент окончания обслуживания всех требований множества $N$ при расписании $\pi$ обозначим как
$$C_{\max}(\pi,t) = \max\limits_{j \in N} C_j(\pi,t).$$

\begin{figure}[h!]
\center{\includegraphics[width=16cm]{Lmax_ab_definitions.png}}
\caption{Параметры требования $j$.}
\label{Lmax_ab:ris:1}
\end{figure}

\begin{problem}\label{Lmax_ab:pr:1}
Для заданного множества требований $N$ и момента времени $t$ построить расписание $\pi^* \in \Pi(N,t)$, при котором
\begin{equation*}\label{Lmax_ab:GF}
    L_{\max}(\pi^*,t) = \min\limits_{\pi \in \Pi(N,t)}L_{\max}(\pi,t).
\end{equation*}
\end{problem}

Данная задача в \cite{GLLR:79} обозначается как $1|r_j|L_{\max}$ и является классической задачей теории расписаний. В \cite{LRB:77} показано, что общий случай задачи $1|r_j|L_{\max}$ является $NP$-трудным в сильном смысле. С момента постановки задачи был выделен ряд полиномиально разрешимых случаев. В \cite{Jack:55} было доказано, что в случае $r_j = 0, j \in N$, решением задачи является расписание, при котором требования упорядочены по неубыванию директивных сроков. Такое расписание также будет оптимальным для случая, когда моменты поступления и директивные сроки согласованы следующим образом: $r_i \leq r_j \Leftrightarrow d_i \leq d_j, \forall i,j \in N$. В случае, когда $d_j = d$ для всех $j \in N$, оптимальное расписание также может быть построено за $O(n \log n)$ операций, по неубыванию моментов поступления. Полиномиальный алгоритм трудоемкости $O(n^2 \log n)$ для случая равенства времен обслуживания требований $p_j = p$ для всех $j \in N$ был представлен в \cite{Sim:78}. В \cite{Hog:96} предложил полиномиальный алгоритм (трудоемкости $O(n^2 \log n)$ операций) для специального случая, когда параметры всех требований $j \in N$ для некоторой константы $A$ удовлетворяют ограничениям
$$d_j - p_j - A \leq r_j \leq d_j - A, \forall j \in N.$$
Полиномиальный алгоритм трудоемкости $O(n^3 \log n)$ операций для случая, когда параметры требований удовлетворяют системе линейных ограничений
\begin{equation*}
\begin{cases}
    d_1 \leq \dots \leq d_n;\\
    d_1 - p_1 - r_1 \geq \dots \geq d_n -  p_n - r_n
\end{cases}
\end{equation*}
был представлен в работе \cite{Laz:08}.

\subsection{Свойства задачи}
Рассматривается случай, когда параметры требований множества $N$ для некоторых действительных чисел $\alpha \in [0, 1]$ и $\beta \in [0, +\infty)$ удовлетворяют системе неравенств
\begin{equation} \label{Lmax_ab:trivial1}
\begin{cases}
    d_1 \leq \dots \leq d_n;\\
    d_1 - \alpha p_1 - \beta r_1 \geq \dots \geq d_n - \alpha p_n - \beta r_n.\\
\end{cases}
\end{equation}
Напомним, $t$ -- момент времени, начиная с которого прибор доступен для обслуживания требований. Выберем из множества $N$ два требования $f = f(N,t)$ и $s = s(N,t)$, такие что
$$f(N,t) = \arg \min\limits_{j \in N} \{d_j | r_j(t) = \min\limits_{i \in N} r_i(t)\},$$
$$s(N,t) = \arg \min\limits_{j \in N \setminus f } \{d_j | r_j(t) = \min\limits_{i \in N \setminus f } r_i(t)\}.$$
В случае, когда $N = \emptyset$, т.е. $|N| = 0$, для любого действительного $t$ положим
$$f(\emptyset, t) = 0, s(\emptyset, t) = 0.$$
В случае, когда $N = \{i\}$, т.е. $|N| = 1$, для любого действительного $t$ положим
$$f(N, t) = i, s(N, t) = 0.$$

Через $(i \rightarrow j)_{\pi}$ будем обозначать, что требование $i$ обслуживается при расписании $\pi$ раньше требования $j$.
\begin{lemma} \label{Lmax_ab:lm:1}
Если для требований множества $N$ выполнены условия (\ref{Lmax_ab:trivial1}) для некоторых $\alpha \in [0,1]$ и $\beta \in [0, +\infty)$, тогда при любом расписании $\pi \in \Pi(N,t)$ для любого требования $j \in N \setminus \{ f\}$ такого, что $(j \rightarrow f)_{\pi}$, верно
\begin{equation}
L_j(\pi,t) < L_f(\pi,t) \label{Lmax_ab:trivial2}
\end{equation}
и для любого требования $j \in N \setminus \{ f,s \}$ такого, что $(j \rightarrow s)_{\pi}$, верно
\begin{equation}
L_j(\pi,t) < L_s(\pi,t), \label{Lmax_ab:trivial3}
\end{equation}
где $f = f(N, t)$, $s = s(N, t)$.
\end{lemma}
\begin{proof}
Для всех работ $j$ таких, что $(j \rightarrow f)_{\pi}$, выполняется неравенство
$$C_j(\pi,t) \leq C_f(\pi,t) - p_f.$$
В случае, когда $d_j \geq d_f$, имеем
$$L_j(\pi,t) = C_j(\pi,t) - d_j < C_f(\pi,t) - d_f = L_f(\pi,t),$$
следовательно, соотношение (\ref{Lmax_ab:trivial2}) выполняется.

Рассмотрим случай, когда для работы $j \in N$, $(j \rightarrow f)_{\pi}$, верно $d_j < d_f$. Из системы (\ref{Lmax_ab:trivial1}) имеем
$$d_j < d_f \Leftrightarrow d_j - \alpha p_j -\beta r_j \geq d_f - \alpha p_f - \beta r_f.$$
Тогда с учетом того, что $r_j > r_f, \alpha \in [0,1], \beta \in [0, \infty)$ и $p > 0$, получаем следующие соотношения:
$$0 \leq \alpha p_j + (1 - \alpha) p_f + \beta (r_j - r_f) \Leftrightarrow \alpha p_f + \beta r_f \leq \alpha p_j + \beta r_j + p_f.$$
C учетом того, что
$$d_j - \alpha p_j -\beta r_j \geq d_f - \alpha p_f - \beta r_f \Leftrightarrow (\alpha p_f + \beta r_f) - (\alpha p_j + \beta r_j) \geq d_f -d_j,$$
получаем $$d_f \leq d_j + p_f.$$
Очевидно, что $ C_j(\pi,t) \leq C_f(\pi,t) - p_f$.
Складывая полученные неравенства, получаем
$$C_{j}(\pi,t) + d_f \leq C_f(\pi,t) + d_j \Leftrightarrow L_j(\pi,t) \leq L_f(\pi,t).$$
Утверждение (\ref{Lmax_ab:trivial2}) доказано.

Доказательство утверждения (\ref{Lmax_ab:trivial3}) проводится аналогично. Достаточно заметить, что работа $s$ из множества $N$ станет работой $f$ во множестве $N \setminus \{f\}$ и необходимо только заменить $N$ на $N \setminus \{f\}$.
\end{proof}

Докажем следующую теорему о свойствах работ $f$ и $s$.
\begin{theorem} \label{Lmax_ab:th:1}
Пусть все работы подмножества $N' \subseteq N$  удовлетворяют системе неравенств (\ref{Lmax_ab:trivial1}) для некоторых $\alpha \in [0,1]$ и $\beta \in [0, +\infty)$. Тогда для любого момента времени $t' \geq t$ и любого расписания $\pi \in \Pi(N', t')$ существует такое расписание $\pi' \in \Pi(N', t')$, что
\begin{equation}
    \begin{cases}
        L_{\max}(\pi ', t') \leq L_{\max} (\pi, t'),\\
        C_{\max}(\pi ', t') \leq C_{\max} (\pi, t')
    \end{cases}
\end{equation}
и в расписании $\pi'$ первой выполняется либо работа $f = f(N', t')$, либо $s = s(N', t').$ Если $d_f \leq d_s$, то первой в расписании $\pi'$ обслуживается работа $f$.
\end{theorem}
\begin{proof}
Пусть  $\pi = (\pi_1, f, \pi_2, s, \pi_3)$, где $\pi_1$, $\pi_2$, $\pi_3$ - частичные подрасписания $\pi$. Рассмотрим расписание $\pi' = (f, \pi_1, \pi_2, s, \pi_3)$. Из определений $r_j(t)$ и $f(N,t)$ для каждого требования $j\in N'$ имеем
$$r_f(t')\leq r_j(t').$$
Следовательно,
$$C_{\max}((f,\pi_1),t') \leq C_{\max}((\pi_1,f),t'),$$
$$C_{\max}(\pi',t') \leq C_{\max} (\pi,t'),$$
поэтому
$$L_j(\pi',t') \leq L_j(\pi,t'), \forall j \in (\pi_2,s,\pi_3).$$
Из леммы \ref{Lmax_ab:lm:1} получаем, что для любого $j \in \{\pi_1\} \cup \{\pi_2\}$ выполняется
$$ L_j(\pi',t') \leq L_s(\pi,t').$$
Для $f$, очевидно, имеем
$$L_f(\pi',t') \leq L_f(\pi,t').$$
Из этих утверждений следует
\begin{equation*}
    \begin{cases}
        L_{\max}(\pi ', t') \leq L_{\max} (\pi, t'),\\
        C_{\max}(\pi ', t') \leq C_{\max} (\pi, t').
    \end{cases}
\end{equation*}
Пусть $\pi = (\pi_1,s,\pi_2,f,\pi_3)$, т.е. работа $s$ выполняется до работы $f$. В этом случае строим расписание $\pi' = (s, \pi_1, \pi_2, f, \pi_3)$  и повторяем доказательство аналогично приведенному выше. Первая часть теоремы доказана.

Положим $d_f \leq d_s$ и $\pi = (\pi_1,s,\pi_2,f,\pi_3)$. Рассмотрим расписания $\pi' = (s,\pi_1,\pi_2,f,\pi_3)$ и $\pi'' = (f,\pi_1,\pi_2,s,\pi_3).$
Тогда для расписаний $\pi, \pi'$ и $\pi''$ будет верно следующее неравенство:
$$C_{\max}((f,\pi_1,\pi_2,s),t') \leq C_{\max} ((s,\pi_1,\pi_2,f), t'),$$
так как $r_f(t') \leq r_s(t')$. Следовательно,
$$L_{\max}(\pi_3,C_{\max}((f,\pi_1,\pi_2,s),t')) \leq L_{\max}(\pi_3,C_{\max}((s,\pi_1,\pi_2,f),t')),$$
поэтому максимум целевой функции $L_{\max}((f,\pi_1,\pi_2,s),t')$ достигается не на требовании $f$.
Максимум целевой функции $L_{\max}((s,\pi_1,\pi_2,f), t')$ не может достигаться на требовании $s$, так как $d_f \leq d_s$ и $C_s((s,\pi_1,\pi_2,f), t') < C_f((s,\pi_1,\pi_2,f), t')$. Тогда согласно лемме \ref{Lmax_ab:lm:1}
$$L_{\max}((f,\pi_1,\pi_2,s),t') = L_s((f,\pi_1,\pi_2,s),t') = C_{\max}((f,\pi_1,\pi_2,s),t') - d_s$$
и
$$L_{\max}((s,\pi_1,\pi_2,f),t') = L_f((s,\pi_1,\pi_2,f),t') = C_{\max}((s,\pi_1,\pi_2,f),t') - d_f.$$
Таким образом, из того, что $d_f \leq d_s$ и $C_{\max}((f,\pi_1,\pi_2,s), t') \leq C_{\max}((s,\pi_1,\pi_2,f), t')$, получаем
$$L_{\max}((f,\pi_1,\pi_2,s),t') \leq L_{\max}((s,\pi_1,\pi_2,f),t')$$
и
$$L_{\max}(\pi'', t') \leq L_{\max}(\pi',t'),$$
что и требовалось доказать.
\end{proof}

Будем называть расписание $\pi' \in \Pi(N,t)$ \textit{эффективным}, если не существует такого расписания $\pi \in \Pi(N,t)$, что выполняется система неравенств
\begin{equation*}
    \begin{cases}
        L_{\max}(\pi, t) \leq L_{\max} (\pi', t),\\
        C_{\max}(\pi, t) \leq C_{\max} (\pi', t),
    \end{cases}
\end{equation*}
причем хотя бы одно из этих неравенств строгое. Тогда если для работ множества $N$ система неравенств (\ref{Lmax_ab:trivial1}) верна при некоторых $\alpha \in [0,1]$ и $\beta \in [0, +\infty)$, то из теоремы \ref{Lmax_ab:th:1} следует, что существует эффективное расписание $\pi'$, при котором либо работа $f = f(N, t)$, либо $s = s(N, t)$ выполняется первой. Более того, если $d_f \leq d_s$, то существует оптимальное расписание, при котором первой выполняется работа $f$.

Пусть $\Omega(N,t)$ -- подмножество множества $\Pi(N,t)$. Расписание $\pi = (i_1, \dots, i_n)$ принадлежит $\Omega (N,t)$, если любая работа $i_k$,
$k = 1,\dots, n$, выбрана из
$$f_k = f(N_{k-1},C_{i_{k-1}}(\pi,t))$$
и
$$s_k = s(N_{i_{k-1}},C_{i_{k-1}}(\pi,t)),$$
где $N_{k-1}$ = $N \setminus \{i_1,\dots,i_{k-1}\} $, $N_0 = N$ и $C_{i_0}(\pi,t) = t$.
Если $d_{f_{k}} \leq d_{s_k}$, то $i_k = f_k$. Если $d_{f_{k}} > d_{s_k}$, тогда либо $i_k = f_k$, либо $i_k = s_k$.
Так как на каждое место в расписании претендует не более двух работ, следовательно, множество $\Omega(N,t)$ содержит не более чем $2^n$ расписаний.
Согласно теореме \ref{Lmax_ab:th:1} всегда можно построить эффективное расписание, которое принадлежит множеству $\Omega(N,t)$, перебрав не более чем $2^n$ вариантов.

Пусть $\omega(N,t)$ -- частичное расписание максимальной длины такое, что при последовательном рассмотрении требований имеем $d_f \leq d_s$. С помощью алгоритма \ref{Lmax_ab:alg:1} для любого набора работ $N$ и времени $t$ можно построить расписание $\omega(N,t)$.

\begin{algorithm}[H]
\NoCaptionOfAlgo
\caption{\textbf{Алгоритм 1}\label{Lmax_ab:alg:1}}
\small
\SetAlgoLined
\KwData{$N, t$}
\KwResult{$\omega(N,t)$}
$N':=N$;\\
$t' := t$;\\
$f:=f(N',t');$\\
$s:=s(N',t');$\\
\eIf{$d_f \leq d_s$}{
    $\omega = (\omega, f);$
}{
    $return(\omega);$
}
$N':=N' \setminus f;$\\
$t':=r_f(t') + p_f;$\\
\eIf{$N \neq \emptyset$}{
    \textit{go to step 3;}
}{
    $return(\omega);$
}
\end{algorithm}
\normalsize

Алгоритм \ref{Lmax_ab:alg:1} заключается в следующем: на каждом проходе цикла 5-13 рассматриваются требования $f(N',t')$ и $s(N',t')$. В случае, если $d_f \leq d_s$, требование $f(N', t')$ при расписании $\omega$ обслуживается с момента времени $r_f(t')$ до момента времени $r_f(t')+p_f$. Выполняется исключение требования $f(N', t')$ из множества $N'$, изменение $t':=r_f(t') + p_f$, после чего выполнение цикла повторяется. Если же $d_f > d_s$, то алгоритм прерывает свою работу и выводит построенное частичное расписание $\omega(N, t)$ (рисунок \ref{Lmax_ab:ris:2}). В случае, когда $d_f > d_s, f = f(N, t), s=s(N,t)$, то $\omega(N,t) = \emptyset$.

\begin{figure}[h!]
\center{\includegraphics[width=16cm]{Lmax_ab_makeomega.png}}
\caption{Построение расписания $\omega(N,t)$.}
\label{Lmax_ab:ris:2}
\end{figure}

\begin{lemma} \label{Lmax_ab:lm:2}
Трудоемкость построения частичного расписания $\omega(N, t)$ алгоритмом \ref{Lmax_ab:alg:1} для любых $N$ и $t$ составляет не более чем $O(n \log n)$ операций.
\end{lemma}
\begin{proof}
На шагах 3-4 алгоритмa \ref{Lmax_ab:alg:1} находятся две работы $f(N', t')$ и $s(N', t')$. Так как работы отсортированы в соответствии с моментами поступления $r_j$, то для нахождения работ $f$ и $s$ потребуется не более чем $O(\log n)$ операций. Ввиду того, что количество проходов цикла 3-13 ограничено мощностью множества $N$, получаем, что для построения частичного расписания $\omega(N, t)$ потребуется не более чем $O(n \log n)$ операций.
\end{proof}

\begin{lemma} \label{Lmax_ab:lm:3}
Если работы множества $N$ удовлетворяют условиям (\ref{Lmax_ab:trivial1}) для некоторых $\alpha \in [0,1]$ и $\beta \in [0, +\infty)$, то любое расписание $\pi \in \Omega(N, t)$ начинается с частичного расписания $\omega(N, t)$.
\end{lemma}
\begin{proof}
Если $\omega(N, t) = \emptyset$, то условие леммы выполняется ввиду того, что любое расписание начинается с пустого расписания.
Если же $\omega(N, t) = (i_1, \dots, i_l)$, то для любого $k = 1, \dots, l$ выполнено $d_{f_k} \leq d_{s_k}$, где $f_k = f(N_{k-1}, C_{k-1}(\pi,t))$ и $s_k = s(N_{k-1}, C_{k-1}(\pi,t)).$ В то же время для $f = f(N_l, C_l(\pi,t))$ и $s = s(N_l, C_l(\pi,t))$ имеем $d_f > d_s$. Ввиду полученных соотношений между директивными сроками и определением $\Omega(N, t)$ получаем, что любое расписание из $\Omega(N, t)$ начинается с $\omega(N, t)$, что и требовалось доказать.
\end{proof}

Будем обозначать: $\omega^1(N,t) = (f(N,t),\omega(N \setminus f,t'))$ и $\omega^2(N,t) = (s(N,t),\omega(N \setminus s,t''))$, где $t' = r_f(t) + p_f$ и $t'' = r_s(t) + p_s$.
Заметим, что из определения $\omega^1(N, t)$ и $\omega^2(N, t)$ и леммы \ref{Lmax_ab:lm:2} следует, что для их нахождения также потребуется $O(n \log n)$ операций.

\begin{corollary} \label{Lmax_ab:cr:1}
Если работы множества $N$ удовлетворяют условиям (\ref{Lmax_ab:trivial1}) для некоторых $\alpha \in [0,1]$ и $\beta \in [0, +\infty)$, то любое расписание $\pi \in \Omega(N, t)$ начинается либо с $\omega^1(N,t)$, либо с $\omega^2(N,t)$.
\end{corollary}
\subsection{Задача на быстродействие с ограничением на максимальное временн\'{о}е смещение}
\begin{problem}\label{Lmax_ab:pr:2}
Упорядочить множество требований $N$ с момента времени $t$ так, чтобы максимальное временн\'{о}е смещение не превышало значения $y$.
Требуется найти оптимальное расписание, удовлетворяющее
\begin{equation*}
\min\limits_{\pi \in \Pi(N, t)} C_{\max}(\pi,t) | L_{\max}(\pi,t) \leq y.
\end{equation*}
Расписание, удовлетворяющее данной целевой функции, обозначим через $\Theta(N, t, y)$. В случае, если не существует такого расписания $\pi \in \Pi(N, t)$, будем говорить, что $\Theta(N, t, y) = \emptyset$.
\end{problem}

Представим алгоритм построения расписания $\Theta(N, t, y)$.
\begin{algorithm}[H]
\NoCaptionOfAlgo
\caption{\textbf{Алгоритм 2}\label{Lmax_ab:alg:2}}
\small
\SetAlgoLined
\KwData{$N, t, y$}
\KwResult{$\Theta(N,t,y)$}
$\Theta:=\omega(N,t);$\\
\If{$L_{\max}(\omega(N,t),t) > y$}{
    $return(\emptyset);$
}
\While{1}{
$N':=N\setminus \Theta;$\\
$t':=C_{\max}(\Theta,t);$\\
\If{$N' =  \emptyset$}{
    $return(\Theta);$
}
\eIf{$L_{\max}(\omega^1(N', t'), t') \leq y$}{
    $\Theta := (\Theta, \omega^1(N', t'));$
}{
    \eIf{$L_{\max}(\omega^2(N', t'), t') \leq y$}{
        $\Theta := (\Theta, \omega^2(N', t'));$
    }{
        $return(\emptyset);$
    }
}
}
\end{algorithm}
\normalsize
На первом шаге алгоритма выполняется построение частичного расписания $\omega(N,t)$, включение его в $\Theta(N,t,y)$, изменение $N':=N' \setminus \Theta$ и $t':=C_{\max}(\Theta,t)$. Далее выполняется построение частичного расписания $\omega^1(N',t')$ и проверка выполнения ограничения $L_{\max} (\omega^1(N',t'),t') \leq y$. В случае положительного результата $\omega^1(N',t')$ добавляется к расписанию $\Theta(N,t,y)$. Затем происходит изменение $N'$ и $t'$, после чего цикл 5-20 повторяется. В противном случае выполняется построение частичного расписания $\omega^2(N',t')$ и проверка выполнения ограничения $L_{\max} (\omega^2(N',t'),t') \leq y$. В случае положительного результата $\omega^2(N',t')$ добавляется к расписанию $\Theta(N,t,y)$, выполняются изменения $N'$ и $t'$, после чего процедура повторяется (рисунок \ref{Lmax_ab:ris:3}). Алгоритм прерывает свою работу, если все требования множества $N$ успешно включены в расписание $\Theta$ или если на каком-то шаге оба расписания $\omega^1(N', t')$ и $\omega^2(N', t')$ не удовлетворяют ограничению на максимальное временн\'{о}е смещение. В этом случае алгоритм возвращает $\Theta(N,t,y) = \emptyset$.

\begin{figure}[h!]
\center{\includegraphics[width=16cm]{Lmax_ab_maketheta.png}}
\caption{Построение расписания $\Theta$.}
\label{Lmax_ab:ris:3}
\end{figure}

\begin{lemma}\label{Lmax_ab:lm:4}
Трудоемкость алгоритма \ref{Lmax_ab:alg:2} не превосходит $O(n^2 \log n)$ операций.
\end{lemma}
\begin{proof}
На шагах 1, 11 и 14 алгоритма \ref{Lmax_ab:alg:2} выполняется построение расписаний $\omega^1(N', t')$ и $\omega^2(N', t')$ с помощью алгоритма \ref{Lmax_ab:alg:1}. Для этого требуется не более чем $O(n \log n)$ операций. В результате прохода цикла 4 - 19 добавляется хотя бы одно требование к расписанию $\Theta$ или возвращается $\Theta = \emptyset$, следовательно, цикл 4-19 повторяется не более чем $n$ раз. Поэтому алгоритм \ref{Lmax_ab:alg:2} находит расписание $\Theta(N, t, y)$ не более чем за $O(n^2 \log n)$ операций.
\end{proof}

Докажем теорему о свойствах построенного расписания $\Theta(N,t,y)$.
\begin{theorem} \label{Lmax_ab:th:2}
Пусть для работ множества $N$ выполнены условия (\ref{Lmax_ab:trivial1}) для некоторых $\alpha \in [0,1]$ и $\beta \in [0, +\infty)$. Тогда если расписание $\Theta(N, t, y)$, найденное с помощью алгоритма \ref{Lmax_ab:alg:2}, не пустое, то для любого расписания $\pi \in \Pi(N,t)$, удовлетворяющего ограничению $L_{\max}(\pi,t) \leq y$, выполнено
$$C_{\max}(\Theta(N,t,y),t) \leq C_{\max}(\pi,t).$$
Если же $\Theta(N, t, y) = \emptyset$, то для любого расписания $\pi \in \Pi(N,t)$ верно
$$L_{\max}(\pi,t) > y.$$
\end{theorem}
\begin{proof}
Так как для любого $\pi \in \Pi(N,t)$ выполнены условия (\ref{Lmax_ab:trivial1}), то по теореме \ref{Lmax_ab:th:1} существует такое расписание $\pi' \in \Omega(N, t)$, для которого
\begin{equation*}
    \begin{cases}
        L_{\max}(\pi ', t) \leq L_{\max} (\pi, t),\\
        C_{\max}(\pi ', t) \leq C_{\max} (\pi, t).
    \end{cases}
\end{equation*}
Из построения расписания $\Theta(N,t,y)$ следует, что оно принадлежит множеству $\Omega(N, t)$. Тогда из леммы \ref{Lmax_ab:lm:3} следует, что расписание $\Theta(N,t,y)$ будет начинаться с частичного расписания $\omega(N, t)$. Обозначим $\Theta_0 = \omega(N, t)$.

После $k$ проходов цикла 4-19 получим частичное расписание $\Theta_k$, при этом $N' = N \setminus \{ \Theta_k \}$ и $t' = C_{\max}(\Theta_k,t)$. Допустим, что существует расписание $\Theta$ с минимальным значением $C_{\max}$, начинающееся с частичного расписания $\Theta_k$ и удовлетворяющее ограничению $L_{\max}(\Theta_k,t) \leq y$. Тогда по лемме \ref{Lmax_ab:lm:3} можно продолжить $\Theta_k$ расписанием из множества $\Omega(N', t')$. Возможны три случая.
\begin{enumerate}
\item[1.] Пусть $\Theta_{k+1} = (\Theta_k,\omega^1(N', t'))$, т.е. $L_{\max}(\omega^1(N',t'),t') \leq y$, тогда $\omega^1(N',t')$ -- частичное расписание с наименьшим значением $C_{\max}$ среди всех возможных продолжений расписания $\Theta_k$, удовлетворяющих $L_{\max}(\Theta_{k+1}, t) \leq y$.\\
\item[2.] Если $\Theta_{k+1} = (\Theta_k,\omega^2(N',t'))$, то
\begin{equation*}
    \begin{cases}
        L_{\max}(\omega^1(N',t'),t') > y,\\
        L_{\max}(\omega^2(N',t'),t') \leq y.
    \end{cases}
\end{equation*}
Это следует из того, что любое расписание из множества $\Omega(N',t')$ может начинаться либо с $\omega^1(N', t')$, либо с $\omega^2(N',t')$ и $L_{\max}(\omega^1(N', t'),t') > y$. Из шагов 11-13 следует, что данный случай возможен, только если $\omega^2(N',t')$ -- единственное возможное продолжение расписания $\Theta_k$.
\item[3.] Рассмотрим теперь случай, когда после $k$ прохода цикла 4-19  имеем $L_{\max}(\omega^1(N', t'),t') > y$ и $L_{\max}(\omega^2(N', t'), t') > y$. Из предположения следует, что если расписание $\Theta \in \Omega(N, t)$ существует, то оно обязательно должно начинаться с $\Theta_k$. Кроме того, для любого $\pi \in \Pi(N',t')$ всегда существует $\pi' \in \Omega(N',t')$ такое, что либо
    $$L_{\max}(\pi,t') \geq L_{\max}(\pi', t') \geq L_{\max}(\omega^1(N', t'),t') > y,$$
    либо
    $$L_{\max}(\pi,t') \geq L_{\max}(\pi', t') \geq L_{\max}(\omega^2(N', t'),t') > y.$$
    Следовательно, $\Theta = \emptyset$.
\end{enumerate}
Таким образом, после не более чем $n$ проходов цикла 4-19 будет построено искомое расписание $\Theta(N, t, y)$. Если хоть раз возникнет случай 3, расписания $\Theta(N, t, y)$ не существует. Что и требовалось доказать.
\end{proof}

\subsection{Алгоритм построения множества оптимальных по Парето расписаний по критериям $C_{\max}$ и $L_{\max}$}
Ниже представлен алгоритм, в результате работы которого для любого набора работ $N$ и момента времени $t$ может быть получено Парето-множество расписаний $\Phi(N,t)$ такое, что $1 \leq | \Phi(N,t) | \leq n$. В случае, когда выполняется условие \ref{Lmax_ab:trivial1}, будет построено оптимальное Парето-множество.

Процесс работы алгоритма \ref{Lmax_ab:alg:3} заключается в следующем.
Из леммы \ref{Lmax_ab:lm:3} следует, что любое оптимальное по критерию $L_{\max}$ расписание из $\Omega(N,t)$  начинается с $\omega(N, t)$, поэтому
обозначим $\pi_0 = \omega(N, t)$ и рассмотрим работу цикла 5-40. Пусть после первых $k$ проходов цикла 5-40 алгоритма \ref{Lmax_ab:alg:3} будет построено частичное расписание $\pi^* = \pi_k$ и множество $\Phi = \{\pi_1', \dots, \pi_m'\}$. Пусть $N' = N \setminus \{\pi_k\}$ и $t' = C_{\max}(\pi_k,t)$ -- значения, полученые на шагах 5 и 6 во время $(k+1)$-го прохода цикла 5-40.
Рассмотрим возможные продолжения расписания $\pi_k$.
\begin{itemize}
\item[1.] Если $L_{\max}(\omega^1(N', t'),t') \leq L_{\max}(\pi_k,t),$ то выполняется присвоение $\pi^* := \pi_{k+1} =(\pi_k, \omega^1(N', t'))$ и возврат на шаг 5 после выбора оптимального продолжения по критерию $C_{\max}$ без нарушения текущего значения целевой функции $L_{\max}(\pi_{k+1},t) \leq L_{\max}(\pi_k,t)$.
\item[2.] $L_{\max}(\pi_k,t) < L_{\max}(\omega^1(N', t'),t') \leq y$ и расписание $\Theta(N', t', L_{\max}(\omega^1(N', t'), t')) = \emptyset$. В этом случае выполняется присвоение $\pi^* := \pi_{k+1} = (\pi_k, \omega^1(N', t'))$ и возврат на шаг 5 после выбора оптимального продолжения по критерию $C_{\max}$ без нарушения ограничения $L_{\max}(\pi_{k+1},t) \leq y$.
\item[3.] $L_{\max}(\pi_k,t) < L_{\max}(\omega^1(N', t'),t') \leq y$ и $\Theta(N', t', L_{\max}(\omega^1(N', t'), t')) \neq \emptyset$. Выполняется присвоение $\pi' := (\pi^*, \Theta(N', t', L_{\max}(\omega^1(N', t'), t')))$. Так как $y' \leq y$, то расписание $\pi'$ удовлетворяет ограничению $L_{\max}(\pi_{k+1},t) \leq  y$. Если значение $C_{\max}(\pi',t)$  увеличилось по сравнению с $C_{\max}(\pi_m',t)$, то увеличивается счетчик $m:=m+1$, выполняется включение $\pi'$ во множество $\Phi$ и изменение ограничения $y$ (шаги 23-26). Если же имеем $C_{\max}(\pi',t) \leq C_{\max}(\pi_m',t)$, то заменяем расписание $\pi_m'$ на $\pi'$ во множестве $\Phi$ (шаг 28). После любого из возможных исходов выполняется возврат на шаг 5.
\item[4.] $L_{\max}(\omega^1(N', t'),t') > y $, $L_{\max}(\omega^2(N', t'),t') \leq y.$ Выполняется присвоение $\pi^* := \pi_{k+1} = (\pi_k, \omega^2(N', t'))$ -- единственный возможный вариант продолжения $\pi_k$ без нарушения ограничения $L_{\max}(\pi_{k+1},t) \leq  y$ (шаг 33). После чего выполняется переход на шаг 5.
\item[5.] $L_{\max}(\omega^1(N', t'),t') > y $, $L_{\max}(\omega^2(N', t'),t') > y.$ Нет возможности продолжить расписаниe $\pi_k$, не нарушая ограничения $L_{\max}(\pi_{k+1},t) \leq y$. Выполнение алгоритма прерывается (шаг 36).
Алгоритм завершает свою работу, если все требования множества $N$ включены в расписание $\pi^*$ или если нет возможности продолжить расписание $\pi^*$, не нарушая ограничения $y$ (шаг 36).
\end{itemize}
\begin{algorithm}[h!]
\NoCaptionOfAlgo
\caption{\textbf{Алгоритм 3}\label{Lmax_ab:alg:3}}
\small
\SetAlgoLined
\KwData{$N, t$}
\KwResult{$\Phi(N,t)$}
$y:=+\infty;$\\
$\pi^*:= \omega(N,t);$\\
$\Phi:=\emptyset;$\\
$m:=0;$\\
\While{1}{
$N':=N \setminus \pi^*;$\\
$t':=C_{\max}(\pi^*,t);$\\
\If{$N' = \emptyset$}{
    $\Phi:=\Phi \cup \{\pi^*\};$\\
    $return(\Phi);$\\
}
\eIf{$L_{\max}(\omega^1(N', t'),t')  \leq L_{\max}(\pi^*,t)$}{
    $\pi^*:=(\pi^*, \omega^1(N', t'))$;\\
}{
    \eIf{$L_{\max}(\omega^1(N', t'),t') \leq y$}{
        $y' := L_{\max}(\omega^1(N', t'), t');$\\
        $\Theta := \Theta(N', t', y');$\\
        \eIf{$\Theta = \emptyset$}{
            $\pi^* :=(\pi^*, \omega^1(N', t'));$
        }{
            $\pi' :=(\pi^*, \Theta);$\\
            \eIf{($m=0$) or ($C_{\max}(\pi'_m,t) < C_{\max}(\pi',t)$)}{
                $m:=m+1;$\\
                $\pi'_m := \pi';$\\
                $\Phi := \Phi \cup \{\pi'_m\};$\\
                $y:=L_{\max}(\pi'_m, t);$\\
            }{
                $\pi'_m := \pi';$
            }
        }
    }{
        \eIf{$L_{\max}(\omega^2(N', t'), t') \leq y$}{
            $\pi^* :=(\pi^*, \omega^2(N', t'));$
        }{
            $\pi^* := \pi'_m;$\\
            $return(\Phi).$
        }
    }
}
}
\end{algorithm}
\normalsize
\begin{lemma}\label{Lmax_ab:lm:5}
Время работы алгоритма \ref{Lmax_ab:alg:3} не превосходит $O(n^3 \log n)$ операций, а мощность множества $\Phi(N,t)$ не превышает $n$.
\end{lemma}
\begin{proof}
Наиболее трудоемкими операциями при проходе цикла 5-40 являются построения частичных расписаний $\omega^1(N', t')$ и $\omega^2(N', t')$ и $\Theta$. Для нахождения $\omega^1(N', t')$ и $\omega^2(N', t')$ требуется $O(n \log n)$ операций, а для нахождения расписания $\Theta$ -- $O(n^2 \log n)$ операций. Так как частичные расписания $\omega^1(N', t')$ и $\omega^2(N', t')$ состоят не менее чем из одной работы, то при каждом проходе цикла к частичному расписанию $\pi^*$ добавляется не менее одной работы, а во множество $\Phi(N,t)$ включается не более одного расписания. Следовательно, число проходов цикла 5-40 алгоритма \ref{Lmax_ab:alg:3} будет не более чем $n$. Т.о., мощность множества $\Phi(N,t)$ не превосходит $n$ и общее количество операций не превышает $O(n^3 \log n)$.
\end{proof}
\begin{theorem} \label{Lmax_ab:th:3}
Пусть для работ множества $N$ выполнены условия (\ref{Lmax_ab:trivial1}) для некоторых $\alpha \in [0,1]$ и $\beta \in [0, +\infty)$. Тогда расписание $\pi^*$, построенное алгоритмом \ref{Lmax_ab:alg:3}, оптимально по критерию $L_{\max}$.

Для любого расписания $\pi \in \Pi(N,t)$ существует такое $\pi' \in \Phi(N,t)$, что
\begin{equation*}
    \begin{cases}
        L_{\max}(\pi',t) \leq L_{\max}(\pi,t),\\
        C_{\max}(\pi',t) \leq C_{\max}(\pi,t)
    \end{cases}
\end{equation*}
и множество расписаний $\Phi(N, t)$ оптимально по Парето в соответствии с критериями $L_{\max}$ и $C_{\max}$.

\end{theorem}
\begin{proof}
Предположим, что существует расписание $\pi \in \Pi(N,t)$, которое не принадлежит $\Phi(N,t)$ и для которого выполнено хотя бы одно из неравенств
\begin{equation}\label{Lmax_ab:ineq:C}
C_{\max}(\pi,t) < C_{\max}(\pi',t)
\end{equation}
или
\begin{equation}\label{Lmax_ab:ineq:L}
L_{\max}(\pi,t) < L_{\max}(\pi',t)
\end{equation}
для любого расписания $\pi' \in \Phi(N,t)$. Согласно теореме \ref{Lmax_ab:th:1} существует расписание $\pi'' \in \Omega(N, t)$ такое, что
\begin{equation*}
    \begin{cases}
        L_{\max}(\pi'',t) \leq L_{\max}(\pi,t),\\
        C_{\max}(\pi'',t) \leq C_{\max}(\pi,t).
    \end{cases}
\end{equation*}
Если $\pi'' \in \Phi(N,t)$, тогда очевидно, что ни одно из условий (\ref{Lmax_ab:ineq:C}) и (\ref{Lmax_ab:ineq:L}) не может быть выполнено. Следовательно, $\pi'' \in \Omega(N,t) \setminus \Phi(N,t)$.

Из определения множества $\Omega(N,t)$ следует, что любое расписание $\pi''$ из множества $\Omega(N,t)$ может быть представлено в виде объединения частичных расписаний $\pi'' = (\omega_0,\omega_1, \dots , \omega_{k''})$, где $\omega_0 = \omega(N,t)$, а $\omega_i$ -- либо $\omega^1(N''_i,C''_i)$, либо $\omega^2(N''_i,C''_i)$ и $N''_i = N \setminus \{\omega_0, \dots , \omega_{i-1}\}$, $C''_i = C_{\max}((\omega_0, \dots , \omega_{i - 1}), t)$, $i = 1, \dots , k''$.

Расписание $\pi'$ имеет аналогичную структуру, т.к. $\Phi(N,t) \subseteq \Omega(N,t)$, т.е.
 $\pi' = (\omega'_0,\omega'_1, \dots , \omega'_{k'})$, где $\omega'_0 = \omega(N,t)$, а $\omega'_i$ -- либо $\omega^1(N'_i,C'_i)$, либо $\omega^2(N'_i,C'_i)$ и $N'_i = N \setminus \{\omega'_0, \dots , \omega'_{i-1}\}$, $C'_i = C_{\max}((\omega'_0, \dots , \omega'_{i - 1}), t)$, $i = 1,  \dots , k'$.

Допустим, что первые $k$ частичных расписаний в $\pi'$ и $\pi''$ совпадают, т.е. $\omega'_i = \omega_i$  $\forall i = 0, \dots, k-1$, и $\omega'_k \neq \omega_k$. Положим, $y = L_{\max}(\omega_0, \dots , \omega_{k-1},t)$, $N_k = N'_k = N''_k$ и $C_k = C'_k = C''_k$. Построим расписание $\Theta = \Theta(N_k,C_k,y)$  с помощью алгоритма \ref{Lmax_ab:alg:2}. Если $\Theta = \emptyset$, то по алгоритму \ref{Lmax_ab:alg:3} имеем: $\omega'_k = \omega^1(N_k,C_k)$. Так как $\omega_k \neq \omega'_k$, получаем, что $\omega_k = \omega^2(N_k,C_k)$. Условие $L_{\max}(\omega^2(N_k,C_k), C_k) \leq y$ не может быть выполнено, так как $\Theta = \emptyset$. Вся структура алгоритма \ref{Lmax_ab:alg:3} построена на том, чтобы расположить работы так плотно, как это возможно, пока не встретится работа с критическим значением $L_{\max}$. Следовательно, продолжая частичное расписание $\omega^1(N_k,C_k)$, имеем
\begin{equation*}
    \begin{cases}
        C_{\max}(\pi',t) \leq C_{\max}(\pi'',t),\\
        L_{\max}(\pi',t) \leq L_{\max}(\pi'',t).
    \end{cases}
\end{equation*}
В том случае, когда $\Theta \neq \emptyset$, для расписания $\pi' = (\omega'_0, \dots, \omega'_k, \Theta)$ имеем
\begin{equation*}
    \begin{cases}
        C_{\max}(\pi',t) \leq C_{\max}(\pi'',t),\\
        L_{\max}(\pi',t) = L_{\max}(\pi'',t).
    \end{cases}
\end{equation*}
Следовательно, для любого расписания $\pi'' \in \Omega(N,t) \setminus \Phi(N,t)$ существует такое расписание $\pi' \in \Phi(N,t)$, что $C_{\max}(\pi',t) \leq C_{\max}(\pi'',t)$ и $L_{\max}(\pi',t) \leq L_{\max}(\pi'',t)$.

Для множества расписаний $\Phi(N,t) = \{\pi'_1, \dots, \pi'_m\}$ имеем (рисунок \ref{Lmax_ab:ris:4}):
\begin{equation}
    \begin{cases}
        C_{\max}(\pi'_1,t) < C_{\max}(\pi'_2,t) < \dots < C_{\max}(\pi'_m,t),\\
        L_{\max}(\pi'_1,t) > L_{\max}(\pi'_2,t) > \dots > L_{\max}(\pi'_m,t).
    \end{cases}
\end{equation}

\begin{figure}[h!]
\center{\includegraphics[width=14cm]{Lmax_ab_Pareto.png}}
\caption{Множество расписаний, оптимальных по Парето.}
\label{Lmax_ab:ris:4}
\end{figure}

Следовательно, $\Phi(N,t)$ -- оптимальное по Парето множество расписаний, причем по лемме \ref{Lmax_ab:lm:5} имеем $|\Phi(N,t)| \leq n$.
Таким образом, получаем, что расписание $\pi'_1$ оптимальное по критерию $C_{\max}$, в то время как расписание $\pi'_m$ имеет наилучшее значение максимального временн\'{о}го смещения $L_{\max}$. Что и требовалось доказать.
\end{proof}

\subsection{Проверка принадлежности примера полиномиально разрешимой области}
Для произвольного примера задачи $1|r_j| L_{\max}$ необходимо знать, можно ли подобрать такие значения $\alpha$ и $\beta$, что система неравенств (\ref{Lmax_ab:trivial1}) будет верной. Для этого необходимо и достаточно решить следующую задачу.
\begin{problem}\label{Lmax_ab:pr:3}
Дано $3n$ действительных чисел: $r_1, \dots, r_n, d_1, \dots, d_n, p_1, \dots, p_n$. Существуют ли такие действительные числа $\alpha \in [0, 1]$ и $\beta \in [0, + \infty)$, что выполняется система неравенств (\ref{Lmax_ab:trivial1})?
\end{problem}

Известно, что $d_1 \leq \dots \leq d_n$, для всех $i=1, \dots, n-1$. Сделаем замены:
$$D_i = d_{i+1} - d_i,$$
$$P_i = p_{i+1} - p_i,$$
$$R_i = r_{i+1} - r_i.$$
Тогда система неравенств (\ref{Lmax_ab:trivial1}) примет вид\\
\begin{equation}{\label{Lmax_ab:trivial4}}
\begin{cases}
    D_1 \geq 0,\\
    \dots\\
    D_{n-1} \geq 0,\\
    \alpha P_1 + \beta R_1 \geq D_1,\\
    \dots\\
    \alpha P_{n-1} + \beta R_{n-1} \geq D_{n-1}.
\end{cases}
\end{equation}

\begin{theorem}\label{Lmax_ab:th:4}
Для набора $r_1, \dots, r_n, d_1, \dots, d_n, p_1, \dots, p_n$ существуют $\alpha_0 \in [0,1]$ и $\beta_0 \in [0, + \infty)$ такие, что система неравенств (\ref{Lmax_ab:trivial4}) выполняется тогда и только тогда, когда существуют $\alpha_1 \in \{0,1\}$ и $\beta_1 \in [0, +\infty)$, для которых система (\ref{Lmax_ab:trivial4}) выполняется.
\end{theorem}
\begin{proof}
Рассмотрим возможные варианты неравенств системы (\ref{Lmax_ab:trivial4}) (рисунок \ref{Lmax_ab:ris:5}). Пусть $M = \{1, \dots, n-1\}$ -- множество индексов, используемых в системе (\ref{Lmax_ab:trivial4}). Представим $M$ в виде объединения подмножеств $M^1 \cup \dots \cup M^7$ в зависимости от значений $P_i$ и $R_i$ в соответствии с правилами, описанными ниже.

\begin{enumerate}
\item[1.] Если для $i \in M$ выполняется $P_i =0, R_i =0$, то $i \in M^1$. Тогда неравенство $\alpha P_i + \beta R_i \geq D_i$ будет выполнено для любых значений $\alpha$ и $\beta$ при $D_i = 0$ и не будет выполнено при $D_i >0$.
\item[2.] Если для $i \in M$ выполняется $P_i =0, R_i \neq 0$, то $i \in M^2$. В этом случае неравенство имеет вид $\beta R_i \geq D_i \Leftrightarrow \beta \geq \frac{D_i}{R_i}$. Обозначим $\min\limits_{i \in M^2} \frac{D_i}{R_i}$ через $\frac{D^2}{R^2}$. Тогда неравенство $\alpha P_i + \beta R_i \geq D_i$ выполняется тогда и только тогда, когда $\beta \geq \frac{D^2}{R^2}$.
\item[3.] Если для $i \in M$ выполняется $P_i \neq 0, R_i = 0$, то $i \in M^3$. В этом случае неравенство имеет вид $\alpha P_i \geq D_i \Leftrightarrow \alpha \geq \frac{D_i}{P_i}$. Обозначим $\min\limits_{i \in M^3} \frac{D_i}{P_i}$ через $\frac{D^3}{P^3}$. Тогда неравенство $\alpha P_i + \beta R_i \geq D_i$ выполняется при всех значениеях $\alpha \geq \frac{D^3}{R^3}$ и только при них.
\item[4.] Если для $i \in M$ выполняется $P_i<0, R_i <0$, то $i \in M^4$. В этом случае решение существует тогда и только тогда, когда $\alpha = \beta = 0$. Следовательно, если $M^4 \neq \emptyset$, то $\alpha_0 = \beta_0 = 0$ -- единственные возможные коэффициенты, при которых (\ref{Lmax_ab:trivial4}) имеет решение.
\item[5.] Если для $i \in M$ выполняется $P_i>0, R_i < 0$, то $i \in M^5$. Тогда
$$P_i + \beta R_i \geq \alpha P_i + \beta R_i \geq D_i,$$
$$\frac{D_i - P_i}{R_i} \geq \beta \geq 0.$$
 Заметим, что неравенство $\alpha P_i + \beta R_i \geq D_i$ будет выполняться при $\alpha = 1$ и $\beta \leq \frac{D_i - P_i}{R_i}$. Обозначим: $B = \min\limits_{i \in M^5} \frac{D_i - P_i}{R_i}$. Заметим, что $B \geq 0$ и для любых $i \in M^5$ неравенство $\alpha P_i + \beta R_i \geq D_i$ будет выполнено при $\alpha = 1$ и $\beta \in [0,B]$.
\item[6.] Если для $i \in M$ выполняется $P_i<0, R_i > 0$, то $i \in M^6$.
\item[7.] Если для $i \in M$ выполняется $P_i > 0, R_i>0$, то $i \in M^7$.
\end{enumerate}

\begin{figure}[h!]
\center{\includegraphics[width=14cm]{Lmax_ab_Case.png}}
\caption{Типы неравенств системы (\ref{Lmax_ab:trivial4}).}
\label{Lmax_ab:ris:5}
\end{figure}

Заметим, что рассмотрены все возможные пары значений $P_i$ и $R_i$, откуда следует, что $M \equiv M^1 \cup \dots \cup M^7$. Предположим, что существуют такие $\alpha_0 \in [0,1]$ и $\beta_0 \in [0, +\infty)$, что система (\ref{Lmax_ab:trivial4}) верна.

Если $M^1 \neq \emptyset$, то для любого $i \in M^1$ верно неравенство $\alpha_0 P_i + \beta_0 R_i \geq D_i \Leftrightarrow 0 \geq D_i$, следовательно, данное неравенство будет выполнено для всех $\alpha \in [0,1]$ и $\beta \in [0, +\infty)$.

Если $M^3 \neq \emptyset$, то для любого $i \in M^3$ неравенство $\alpha P_i + \beta R_i \geq D_i$ верно тогда и только тогда, когда $\alpha \geq \frac{D^3_0}{P^3_0}$. Следовательно, если неравенство имеет решение при некотором $\alpha_0 \in [0,1]$, то при $\alpha_1 = 1 \geq \alpha_0 \geq \frac{D^3_0}{P^3_0}$ данное неравенство также будет верно для всех $i \in M^3$.

Если $M^4 \neq \emptyset$, то единственное возможноe решение системы (\ref{Lmax_ab:trivial4}) -- это $\alpha_0 = 0, \beta_0 = 0.$ Следовательно система (\ref{Lmax_ab:trivial4}) верна при $\alpha_1 = \beta_1 = 0$.

Пусть $M^5 \neq \emptyset$. Покажем, что неравенство $\alpha P_i + \beta R_i \geq D_i$ может быть выполнено для всех $i \in M^2 \cup M^5 \cup M^6 \cup M^7$ при некоторых $\alpha_0 \in [0,1]$ и $\beta_0 \in  [0, +\infty)$ тогда и только тогда, когда оно будет выполнено при $\alpha_1 = 1$ и $\beta_1 = B$. Для $i \in M^5$ данное утверждение очевидно.

Если $M^5 \neq \emptyset$ и $M^2 \neq \emptyset$, то выполняется $\beta_0 \geq \frac{D^2}{R^2}$. Заметим, что так как $M^5 \neq \emptyset$, то $B \geq \beta_0$, следовательно, неравенство $\alpha P_i + \beta R_i \geq D_i$ будет выполнено для всех $i \in M^2 \cup M^5$.

Если $M^5 \neq \emptyset$ и $M^6 \neq \emptyset$, то для любых $i \in M^5$ и $j \in M^6$ выполнены неравенства $\alpha_0 P_i + \beta_0 R_i  - D_i \geq 0$ и $\alpha_0 P_j + \beta_0 R_j  - D_j \geq 0$. Возьмем $i = \arg \min\limits_{i \in M^5} \frac{D_i - P_i}{R_i}$. Имеем
$$\frac{D_i - \alpha_0 P_i}{R_i} \geq \beta_0 \geq \frac{D_j - \alpha_0 P_j}{R_j} \Rightarrow (\frac{D_i}{R_i} - \frac{D_j}{R_j}) - \alpha_0 (\frac{P_i}{R_i} - \frac{P_j}{R_j}) \geq 0.$$
Так как $R_i < 0$ и $R_j >0,$ то
$$\frac{D_i}{R_i} - \frac{D_j}{R_j} < 0 \Rightarrow \alpha_0 (\frac{P_i}{R_i} - \frac{P_j}{R_j}) > 0.$$
следовательно,
$$(\frac{D_i}{R_i} - \frac{D_j}{R_j}) - (\frac{P_i}{R_i} - \frac{P_j}{R_j}) \geq 0,$$
т.е.
$$B = \frac{D_i - P_i}{R_i} \geq \frac{D_j - P_j}{R_j},$$
$$P_j + B R_j \geq D_j.$$
Таким образом, получаем, что для всех $j \in M^6$ неравенство $\alpha P_j + \beta R_j \geq D_j$ будет выполнено при значениях $\alpha_1 = 1$ и $\beta_1 = B$.

Если $M^5 \neq \emptyset$ и $M^7 \neq \emptyset$, то для любых $i \in M^5$ и $j \in M^7$ выполнены неравенства $\alpha_0 P_i + \beta_0 R_i  - D_i \geq 0$ и $\alpha_0 P_j + \beta_0 R_j  - D_j \geq 0$.
Пусть $i = \arg \min\limits_{i \in M^5} \frac{D_i - P_i}{R_i}$. Получаем
$$\frac{D_i - \alpha_0 P_i}{R_i} \geq \beta_0 \geq \frac{D_j - \alpha_0 P_j}{R_j} \Rightarrow (\frac{D_i}{R_i} - \frac{D_j}{R_j}) - \alpha_0 (\frac{P_i}{R_i} - \frac{P_j}{R_j}) \geq 0.$$
Так как $0 \leq \alpha_0 \leq 1$ $\frac{P_i}{R_i} < 0$ и $\frac{P_j}{R_j} > 0$, имеем
$$(\frac{D_i}{R_i} - \frac{D_j}{R_j}) - (\frac{P_i}{R_i} - \frac{P_j}{R_j}) \geq (\frac{D_i}{R_i} - \frac{D_j}{R_j}) - \alpha_0 (\frac{P_i}{R_i} - \frac{P_j}{R_j}) \geq 0.$$
Следовательно,
$$B = \frac{D_i - P_i}{R_i} \geq \frac{D_j - P_j}{R_j},$$
$$P_j + B R_j \geq D_j.$$
Таким образом, для всех $j \in M^7$ неравенство $\alpha P_j + \beta R_j \geq D_j$ будет выполнено при значениях $\alpha_1 = 1$ и $\beta_1 = B$.

Из доказанного выше следует, что если $M^5 \neq \emptyset$, то неравенство $\alpha P_i + \beta R_i \geq D_i$ будет выполнено для всех $i \in M^2 \cup M^5 \cup M^6 \cup M^7$ при $\alpha_1 = 1$ и $\beta_1 = B$. Обратное утверждение очевидно, достаточно взять $\alpha_0 = 1$ и $\beta_0 = B$.

Если же $M^5 = \emptyset,$ то все приведенные рассуждения будут верны для значения $B' = \max\limits_{i \in M^2 \cup M^6 \cup M^7} \frac{D_i - P_i}{R_i}$. Таким образом, если для некоторых $\alpha_0 \in [0,1]$ и $\beta_0 \in [0, +\infty)$ система неравенств (\ref{Lmax_ab:trivial4}) верна, то:
\begin{itemize}
\item если $M^4 \neq \emptyset$, то система (\ref{Lmax_ab:trivial4}) верна при $\alpha_1 = 0$ и $\beta_1 = 0$;
\item если $M^4 = \emptyset, M^5 \neq \emptyset$, система (\ref{Lmax_ab:trivial4}) верна при $\alpha_1 = 1$, $\beta_1 = B$;
\item если $M^4 = \emptyset, M^5 = \emptyset, M \neq M^3$, система (\ref{Lmax_ab:trivial4}) верна при $\alpha_1 = 1$, $\beta_1 = B'$;
\item если $M = M^3$, то система (\ref{Lmax_ab:trivial4}) верна при $\alpha_1 = 1,$ $\beta_1 = 0$.
\end{itemize}

Покажем алгоритм нахождения $\alpha_1 \in \{0,1\}, \beta_1 \in [0, +\infty)$.\\
\begin{algorithm}[H]
\NoCaptionOfAlgo
\caption{\textbf{Алгоритм 4}\label{Lmax_ab:alg:4}}
\small
\SetAlgoLined
\KwData{$P_1, \dots, P_n, R_1, \dots, R_n, D_1, \dots, D_n$}
\KwResult{$\alpha_1, \beta_1$}

\eIf{$M^4 \neq \emptyset$}{
    $\alpha_1 := 0, \beta_1 := 0;$\\
    }{
    \eIf{$M^5 \neq \emptyset$}{
        $\alpha_1 := 1, \beta_1 := \min\limits_{i \in M^5} \frac{D_i - P_i}{R_i}$;\\
    }{
        \eIf{$M \neq M^3$}{
            $\alpha_1 := 1, \beta_1 := \max\limits_{i \in M^2 \cup M^6 \cup M^7} \frac{D_i - P_i}{R_i};$\\
        }{
            $\alpha_1 := 1, \beta_1 := 0;$
        }
    }
}
\For{($i = 1, i<n, i++$)}{
    \If{$\alpha_1 P_i + \beta_1 R_i < D_i$}{
        $return$\textit{(Не существует $\alpha \in [0,1]$ и $\beta \in [0, +\infty)$, для которых система (\ref{Lmax_ab:trivial1}) верна.);}\\
    }
}
$return(\alpha_1, \beta_1)$.\\
\end{algorithm}
\normalsize

Для того чтобы удостовериться, существуют ли $\alpha_0 \in [0,1]$ и $\beta_0 \in [0, +\infty)$ такие, что система (\ref{Lmax_ab:trivial4}) выполняется,
достаточно проверить на выполнимость все неравенства системы для найденных значений $\alpha_1$ и $\beta_1$. Если хотя бы одно из неравенств неверно, то
из доказательства, приведенного выше, следует, что не существует таких $\alpha_0 \in [0,1]$ и $\beta_0 \in [0, +\infty)$, для которых система неравенств (\ref{Lmax_ab:trivial4}), а следовательно, и система (\ref{Lmax_ab:trivial1}) верны.

Таким образом, система неравенств (\ref{Lmax_ab:trivial1}) верна при некоторых $\alpha_0 \in [0,1]$ и $\beta_0 \in [0, +\infty)$ тогда и только тогда, когда будет верна система неравенств (\ref{Lmax_ab:trivial4}) при $\alpha_1, \beta_1$, найденных с использованием алгоритма \ref{Lmax_ab:alg:4}. Что и требовалось доказать.
\end{proof}

\begin{lemma}\label{Lmax_ab:lm:6}
Трудоемкость алгоритма \ref{Lmax_ab:alg:4} не превышает $O(n \log n)$ операций.
\end{lemma}
\begin{proof}
Алгоритм \ref{Lmax_ab:alg:4} выполняет одну сортировку, две операции присвоения, $O(n)$ операций по выяснению типа неравенств и проверку выполнимости $O(n)$ неравенств. Наиболее трудоемкой частью является сортировка сложностью $O(n \log n)$ операций.
\end{proof}

\subsection{Заключение}
Расширена полиномиально разрешимая область классической $NP$ - трудной в сильном смысле задачи $1|r_j|L_{\max}$. Представлен алгоритм построения множества Парето-оптимальных расписаний в соответствии с критериями $L_{\max}$ и $C_{\max}$ трудоемкостью $O(n^3 \log n)$ операций. Представлен алгоритм для определения принадлежности примера к полиномиально разрешимой области и нахождения параметров $\alpha$ и $\beta$ трудоемкостью $O(n \log n)$ операций.
